%%% BEGIN Times font
\usepackage{mathptmx}
\usepackage[scaled=.82]{beramono}
\usepackage[scaled=.86]{helvet}
\DeclareSymbolFont{letters}{OML}{txmi}{m}{it}

%%% END Times font

%\theoremstyle{plain}
%\newtheorem{theorem}{Theorem}%[section]
%\newtheorem{lemma}[theorem]{Lemma}%[section]
%\newtheorem{corollary}[theorem]{Corollary}%[section]

%\theoremstyle{definition}
%\newtheorem{definition}{Definition}%[section]

%\theoremstyle{remark}
%\newtheorem*{proofsketch}{Proof sketch}
%\newtheorem{example}{Example}%[section]

\spnewtheorem*{proofsketch}{Proof sketch}{\itshape}{\rmfamily}

\usepackage{url}
\urldef{\mailsa}\path|firstname.lastname@epfl.ch|
\newcommand{\keywords}[1]{\par\addvspace\baselineskip
\noindent\keywordname\enspace\ignorespaces#1}

\newcommand\cvcd{{\cvc}h\xspace}
\newcommand\cvcf{{\cvc}f\xspace}
\newcommand\cvcfe{{\cvc}fh\xspace}
\newcommand\cvcfm{{\cvc}fm\xspace}
\newcommand\ziiib{\ziii}

\renewcommand\models{\mathrel{\vthinspace\vDash}}

%\newcommand{\DDD}{\mathcalx{D}}
\newcommand\DDD{\Delta}

%\newcommand\return{$\mathsf{return}$\xspace}
\newcommand\return{}

%\newcommand\bigtuple[1]{$(${#1}$)$}
\newcommand\bigtuple[1]{$\bigl(${#1}$\bigr)$}

%%% for \mathcalx
\DeclareFontFamily{OT1}{pzc}{}
\DeclareFontShape{OT1}{pzc}{m}{it}{<-> s * [1.10] pzcmi7t}{}
\DeclareMathAlphabet{\mathcalx}{OT1}{pzc}{m}{it}

%%% for correctly sized Greek
\DeclareSymbolFont{letters}{OML}{txmi}{m}{it}

%\newcommand\pneg{\neg}
\newcommand\pnegate{-}
%\newcommand\pnegate{{\sim}}

\newcommand{\con}[1]{\mathsf{#1}}
\let\const=\con

\renewcommand\vec[1]{\overline{#1}}

\let\oldcup=\cup
\def\cup{\mathrel{\oldcup}}

\let\oldchi=\chi
\def\chi{{\vthinspace\oldchi}}

\let\oldSigma=\Sigma
\def\Sigma{\mathrm{\oldSigma}}

\let\oldDelta=\Delta
\def\Delta{\mathrm{\oldDelta}}

\def\Chi{\mathrm{X}}

\let\oldneg=\neg
\def\neg{\oldneg\;}

\let\oldvee=\vee
\def\vee{\mathrel{\oldvee}}
\let\oldwedge=\wedge
\def\wedge{\mathrel{\oldwedge}}

\newcommand\win{\bf}
%\newcommand\bench{\ttfamily}

\newcommand\isanun{Isa\xspace}
\newcommand\isa{Ip\xspace}
\newcommand\isam{Ip-Mut\xspace}
\newcommand\leon{Leon\xspace}
\newcommand\leonm{Leon-Mut\xspace}

%\newcommand{\cvc}{\textsc{cvc}{\small 4}\xspace}
%\newcommand{\cvciii}{\textsc{cvc}{\small 3}\xspace}
%\newcommand{\ziii}{\textsc{z}{\small 3}\xspace}

%%% @ANDY, @CESARE: I hope you'll forgive me these. I find the above fonts
%%% distracting. --JB
\newcommand\cvc{CVC4\xspace}
\newcommand\cvciii{CVC3\xspace}
\newcommand\ziii{Z3\xspace}

\newcommand{\teq}{\approx}
\newcommand{\cc}[1]{#1^*}

%%% I'm trying to avoid bold whenever it can be avoided (except for headings
%%% etc.) --JB
%\newcommand{\terms}{\mathbf{T}}
\newcommand{\terms}{\mathcalx{T}}

%\newcommand{\functerms}{\mathbf{DFT}}
\newcommand{\vars}{\mathbf{V}}
\newcommand{\M}{\mathcalx{M}}
\newcommand{\I}{\mathcalx{I}}
\newcommand{\J}{\mathcalx{J}}
\def\AIF{\qtab\keyword{if}\ }
\def\THEN{\ \keyword{then}\ }
\def\AELSE{\untab\qtab\keyword{else}\ }
\def\FI{\untab}
\def\RETURN{\keyword{return}\ }
\def\ENDPROC{\untab}
%%% @ANDY: I'm trying "A" like "abstraction", again in sync with the abstract--concrete
%%% terminology from abstract interpretation
\newcommand{\conv}{\mathcalx{A}}

%\newcommand{\stypes}[1]{#1^\mathrm{s}}
\newcommand{\stypes}[1]{#1^\mathrm{ty}}

\newcommand{\svars}[1]{#1^\mathrm{v}}
\newcommand{\sfuns}[1]{#1^\mathrm{f}}
\newcommand{\sfundefs}[1]{#1^\mathrm{dfn}}
%\newcommand{\sfunndefs}[1]{#1^\mathrm{nf}}
\newcommand{\typeint}{\ty{Int}}

\newcommand{\pnone}{\con{none}}
\newcommand{\ppos}{\con{pos}}
\newcommand{\pneg}{\con{neg}}
%%% @ANDY: as usual, revert if you don't like
% too confusing, too many different meanings for + and -, with below the latter is a theory symbol, a polarity, and a negation of polarities :)
%\newcommand{\pnone}{{\pm}}
%\newcommand{\ppos}{{+}}
%\newcommand{\pneg}{{-}}
\newcommand{\pol}{\con{pol}}

%\newtheorem{remark}{Remark}

\newcommand\ty[1]{\con{#1}}
\newcommand{\Bool}{\ty{Bool}}
\newcommand{\ltrue}{\top}
\newcommand{\lfalse}{\bot}
\newcommand{\lite}{\con{ite}}

%\newcommand\concret{\con{a}}
%\newcommand\concret{\con{c}}
\newcommand\concret{\gamma} %%% from the abstract interpretation literature

\newcommand{\boolop}{\con{b}}
\newcommand{\forallf}[1]{\forall_{\!#1\:}}
\newcommand{\fnull}{\emptyset}
\newcommand{\vecfarg}[1]{\vec{\concret}_{#1}}
\newcommand{\farg}[1]{\concret_{#1}}
\newcommand{\fargx}[2]{\concret_{#1,#2}}
\newcommand{\fargtype}[1]{\alpha_{#1}}

\newcommand{\Sigmalia}{\Sigma_{\mathcalx{LIA}}}
\newcommand{\extendsig}[1]{\mathcalx{E}( #1 )}

\newcommand{\rem}[1]{\textcolor{red}{[#1]}}
\newcommand{\ajr}[1]{\rem{#1 --ajr}}
\newcommand{\jb}[1]{\rem{#1 --jb}}
\newcommand{\ct}[1]{\rem{#1 --ct}}

\newcommand{\negvthinspace}{\kern-0.083333em}
\newcommand{\vthinspace}{\kern+0.083333em}
\newcommand{\vvthinspace}{\kern+0.0416666em}
\newcommand{\typ}[1]{^{\vthinspace #1}}

\newcommand{\definefunreccmd}{\con{define}\text{-}\con{fun}\text{-}\con{rec}}
\newcommand{\definefunsreccmd}{\con{define}\text{-}\con{funs}\text{-}\con{rec}}

\newcommand{\Mo}{{\mathcal{I}}}

%\newcommand{\euf}{\ensuremath{\mathrm{E}}\xspace}
\newcommand{\euf}{\ensuremath{\mathcalx{UF}}\xspace}
%\newcommand{\ari}{\ensuremath{\mathrm{A}}\xspace}
\newcommand{\ari}{\ensuremath{\mathcalx{A}}\xspace}

\newcommand{\absconstraints}{\mathrm{X}}

%\input{scalalistings}
%\input{smtlib2listings}
\usepackage{program}

\def\squareforqed{\hbox{\rlap{$\sqcap$}$\sqcup$}}

\newcommand\xend{{\hfill$\scriptstyle\blacksquare$}}
\renewcommand\qed{{\hfill$\squareforqed$}}

\hyphenation{counter-example counter-examples Isa-belle}

\begin{document}

\title{Model Finding for Recursive Functions in SMT\thanks{%
This research is partially supported by the Inria technological development
action ``Contre-exemples utilisables par Isabelle et Coq'' (CUIC).
}
}
\titlerunning {Model Finding for Recursive Functions in SMT}

\author {Andrew Reynolds\inst{1} \and Jasmin Christian Blanchette\inst{2,3} \and \\ Simon~Cruanes\inst{2} \and Cesare~Tinelli \inst{1}}
\authorrunning {A. Reynolds, J. C. Blanchette, C. Tinelli}
\institute{
Department of Computer Science, The University of Iowa, USA
\and
Inria Nancy -- Grand Est \& LORIA, Villers-l\`es-Nancy, France
\and
Max-Planck-Institut f\"ur Informatik, Saarbr\"ucken, Germany
%\and
%\'Ecole Polytechnique F\'ed\'erale de Lausanne (EPFL), Switzerland
}

\maketitle

%% The institutions above shouldn't count as footnotes
\setcounter{footnote}{0}

\begin{abstract}
\noindent
SMT solvers have recently been extended with techniques for finding models
in presence of universally quantified formulas in some restricted fragments.
This paper introduces a translation which reduces axioms specifying a large
class of recursive functions, including well-founded (terminating) functions,
to universally quantified formulas for which these
techniques are applicable.
An empirical evaluation confirms that the approach improves 
the performance of existing solvers on benchmarks from two sources. 
The translation is implemented as a preprocessor in the CVC4 solver
and in a new higher-order model finder called Nunchaku.
\end{abstract}

\section{Introduction}
\label{sec:introduction}

%\ct{The paper used "recursive datatype" in some places and "inductive datatype" in others. I have replaced both by "algebraic datatype" which is more accurate for \cvc.
%("recursive datatype" is a more general notion and "inductive datatype" has a stronger meaning in higher-order logics.)}  
%% I agree --JB.

Many solvers based on SMT (satisfiability modulo theories) can reason about
quantified formulas using incomplete instantiation-based methods
\cite{MouraBjoerner07,ReynoldsTinelliMoura14}.
These methods work well in the context of proving (i..e, showing
unsatisfiability), but they are of little help for finding models (i.e.,
showing satisfiability). Often, a single universal quantifier in one of the
axioms of a problem is enough to prevent the discovery of models.

In the past few years, techniques have been developed to find models for
quantified formulas in SMT.
Ge and de Moura \cite{GeDeM-CAV-09} introduced a complete instantiation-based
procedure for formulas in the \relax{essentially uninterpreted} fragment.
This fragment is limited to universally quantified formulas where all
%ct occurrences of 
variables occur as direct subterms of uninterpreted
functions---e.g., $\forall x%\typ{\typeint} --- the type is not important, right?
.\;\, \con{f}( x )
\teq \con{g}( x ) + 5$.
% since all
% occurrences of $x$ occur beneath uninterpreted functions $\con{f}$ and
% $\con{g}$.
Other syntactic criteria extend
this fragment slightly, including cases when variables occur as arguments of
arithmetic predicates. Subsequently, Reynolds et al.\
\cite{ReyEtAl-1-RR-13,reynolds-et-al-2013} introduced techniques for finding finite
models for quantified
formulas over uninterpreted types and types having a fixed finite
interpretation. %, such as fixed-width bit vectors and enumerated datatypes.
These techniques can
find a model for a formula such as $\forall x,\, y : \tau.\;\, x \teq
y \vee \allowbreak \neg \con{f}( x ) \teq \con{f}( y )$, where $\tau$ is an uninterpreted type.
% where the uninterpreted type $\tau$ is
% interpreted as a finite set and $\con{f}$ as injective finite map.

Unfortunately, none of these fragments can accommodate the vast majority of
quantified formulas that correspond to recursive function definitions: The
essentially uninterpreted fragment does not allow the argument of a
recursive function to be used inside a complex term on the right-hand side,
whereas the finite model finding techniques %of Reynolds et al.\
are not applicable for functions over infinite domains such as the integers or
algebraic datatypes. A simple example where both approaches fail is
$\forall x : {\typeint}.\;\, \lite\bigl(
x \leq 0,\allowbreak\; \con{p}( x ) \teq 1,\allowbreak\; \con{p}( x ) \teq 2 * \con{p}( x - 1 ) \bigr)$.
This state of affairs is unsatisfactory, given the frequency of
recursive definitions in practice and the impending addition of a dedicated
command for introducing them, \texttt{define-fun-rec}, to the SMT-LIB standard \cite{smtlib25}.

%In this paper,
We present a method for translating formulas involving recursive function
definitions into formulas where finite model finding techniques can be applied.
The recursive functions must meet a semantic criterion to be admissible
(Section~\ref{sec:prelim}). This criterion is met by well-founded (terminating)
recursive function definitions. \begin{rep}It is not met by inconsistent
definitions such as $\forall x : {\typeint}.\;\, \con{f}(x) \teq \con{f}(x) +
1$.\end{rep} %ct and it is possible to construct consistent examples that do not meet it.
%as well as other example of consistent definition.

We define a translation for a class of
formulas involving admissible recursive function definitions
(Section~\ref{sec:encoding}). The main insight is that a
recursive definition $\forall x : \tau.\;\, \const{f}(x) \teq t$
can be translated to
$\forall a :
\fargtype{\tau}.\;\, \const{f}(\farg{\con{f}}(a)) \teq t[\farg{\con{f}}(a)/x]$, where
$\fargtype{\tau}$ is an uninterpreted \relax{abstract} type and $\farg{\con{f}} \begin{rep}:
\fargtype{\tau} \to \tau\end{rep}$ converts the abstract type to the concrete
type. Additional constraints ensure that the abstract values that are relevant
to the formula's satisfiability
exist. The translation preserves satisfiability and\begin{rep}, for admissible definitions,\end{rep}
unsatisfiability, and makes finite model finding possible for problems in this
class.\begin{conf} Detailed proofs of correctness are included in a technical
report \cite{our-report}.\end{conf}

Our empirical evaluation on benchmarks from the IsaPlanner proof planner~\cite{DBLP:conf/itp/JohanssonDB10}
and the Leon verifier~\cite{blanc2013overview} provides
evidence that this translation improves the effectiveness of the SMT solvers
\cvc and \ziii for finding countermodels to verification conditions
(Section~\ref{sec:evaluation}). The approach is implemented as a preprocessor
in \cvc and in the Nunchaku model finder
(Section~\ref{sec:implementations}). \begin{rep}%
When invoked with a specific
command-line option, the solver assumes that functions introduced using the
\texttt{define-fun-rec} command are admissible and translates them accordingly.\end{rep}

%this option targets users who wish to determine the satisfiability of inputs
%that involving function definitions that are known to be consistent, whereas it
%does not target users who wish to check the consistency of a set of function
%definitions.

%
\section{Preliminaries}
\label{sec:prelim}

Our setting is a monomorphic (or many-sorted) first-order logic
like the one defined by SMT-LIB \cite{smtlib25}.
A \emph{signature} $\Sigma$ consists of
a set $\stypes{\Sigma}$ of %ct type symbols
first-order types
%
%\ct{What the rationale for using the letter $y$ in $\stypes{\Sigma}$?}
%\jb{y as in t\textbf{y}pe. changed to `ty'}
(or sorts) and a set $\sfuns{\Sigma}$ of function symbols over these types.
We assume that signatures always contain a Boolean type $\Bool$ and constants
$\ltrue, \lfalse : \Bool$ for truth and falsity,
an infix equality predicate ${\teq} : \tau \times \tau \to \Bool$
for each $\tau \in \stypes{\Sigma}$,
standard Boolean connectives ($\neg$, $\wedge$, $\vee$, etc.),
and an if--then--else function symbol
$\lite : \Bool \times \tau \times \tau \rightarrow \tau$
for each $\tau \in \stypes{\Sigma}$.
%
For each $\tau \in \stypes{\Sigma}$,
we fix an infinite set $\svars{\Sigma}_\tau$ of \emph{variables of type $\tau$} and
define $\svars{\Sigma}$ as $\bigcup_{\tau \in \stypes{\Sigma}} \svars{\Sigma}_\tau$.
$\Sigma$-terms are built as usual over functions symbols in $\Sigma$ and variables in $\svars{\Sigma}$.
%
Formulas are terms of type $\Bool$.
We write $t\typ{\tau}$ to denote %ct a metavariable ranging over 
terms of type~$\tau$ and $\terms( t )$ to denote the set of subterms in $t$.
Given a term $u$, we write $u[\vec t/\vec x]$ to denote the result of replacing
all occurrences of $\vec x$ with $\vec t$ in $u$.
\begin{rep}When applied to terms, the symbol $=$ denotes syntactic equality.\end{rep}

%We write $\functerms^\Sigma( t )$ to denote the set of $f$-applications in $t$ such that $\con{f} \in \sfundefs{\Sigma}$.

A \emph{$\Sigma$-interpretation $\I$} %is a mathematical structure that
maps each type $\tau \in \stypes{\Sigma}$ to a nonempty set $\tau^\I$,
the \emph{domain} of~$\tau$ in~$\I$,
each function symbol $\con{f} : \tau_1 \times \cdots \times \tau_n \rightarrow \tau$ in $\sfuns{\Sigma}$
to a total function $\con{f}^\I : \tau_1^\I \times \cdots \times \tau_n^\I \rightarrow \tau^\I$,
and each variable $x:\tau$ of $\svars{\Sigma}$ to an element of $\tau^\I$.
A \emph{theory} is a pair $T = (\Sigma, \Mo)$ where
$\Sigma$ is a signature and $\Mo$ is a class of $\Sigma$-interpretations,
the \emph{models} of $T$, closed under variable reassignment
(i.e., for every $I \in \Mo$, every $\Sigma$-interpretation that differs
from $I$ only on the variables of $\svars \Sigma$ is also in $\Mo$).
A $\Sigma$-formula $\varphi$ is \emph{$T$-satisfiable}
if it is satisfied by some interpretation in $\Mo$\begin{rep};
otherwise, it is \emph{$T$-unsatisfiable}\end{rep}.
A formula $\varphi$ \emph{$T$-entails} $\psi$, written $\varphi \models_T \psi$,
if all interpretations in $\Mo$ that satisfy $\varphi$ also satisfy $\psi$.
Two formulas $\varphi$ and $\psi$ are \emph{$T$-equivalent} 
if each $T$-entails the other.
If $T_1 = (\Sigma_1, \Mo_1)$ is a theory and $\Sigma_2$ is a signature 
with $\sfuns{\Sigma_1} \cap \sfuns{\Sigma_2} = \emptyset$,
the \emph{extension of $T_1$ to $\Sigma_2$} is the theory $T = (\Sigma, \Mo)$ where 
$\sfuns{\Sigma} = \sfuns{\Sigma_1} \cup \sfuns{\Sigma_2}$,
$\stypes{\Sigma} = \stypes{\Sigma_1} \cup \stypes{\Sigma_2}$,
and $\Mo$ is the set of all $\Sigma$-interpretations $\I$
whose $\Sigma_1$-reduct is a model of $T_1$.
We refer to the symbols of $\Sigma_2$ that are not in $\Sigma_1$ as \emph{uninterpreted}.
For the rest of the paper, \relax{we fix a theory $T = (\Sigma, \Mo)$ 
with uninterpreted symbols} constructed as above.

Unconventionally,
we consider \emph{annotated quantified formulas} of the form
$\forallf{\const{f}} \vec x.\; \varphi$, where $\con{f} \in \sfuns{\Sigma}$ is
uninterpreted. Their
semantics is the same as for standard quantified formulas $\forall \vec x.\; \varphi$.
Given $\con{f} : \tau_1 \times \cdots \times \tau_n \rightarrow \tau$,
an annotated
quantified formula $\forallf{\con{f}} \vec x.\; \varphi$ is a \emph{function definition}
(\,\emph{for $\con{f}$}\vthinspace) if $\vec x$ is a tuple of variables
$x_1 : \tau_1$, $\ldots,$ $x_n : \tau_n$
and $\varphi$ is a quantifier-free formula 
$T$-equivalent to $\con{f}( \vec x ) \teq t$ for some term $t$ of type $\tau$.
%\ct{Why can't $\varphi$ be existential instead of just quantifier free?} 
%I don't think it makes much sense to have any quantifiers within function bodies, 
%since they aren't operational, that's why I put quantifier-free. --ajr
%%\jb{Actually, in Isabelle we can have them and we use them sometimes. It certainly
%% makes sense. I can give you real examples if you like.}
We write $\exists
\vec x.\; \varphi$ as an abbreviation for $\neg \forall \vec x.\; \neg \varphi$.

\begin{definition}\rm
A formula $\varphi$ is in \emph{definitional form with respect to}
$\{ \con{f}_1, \ldots, \con{f}_n \} \subseteq \sfuns{\Sigma}$ if it is of the
form
%
%\kern-\abovedisplayskip
%\kern+\abovedisplayshortskip
%
$(\forallf{\con{f}_1} \vec x_1.\; \varphi_1) \wedge \cdots \wedge
(\forallf{\con{f}_n} \vec x_n.\; \varphi_n) \wedge \psi$,
%
where $\con{f}_1, \ldots, \con{f}_n$ are distinct\begin{rep} function symbols\end{rep},
%ct added
$\forallf{\con{f}_i} \vec x_i.\; \varphi_i$ is a function definition
for $i = 1, \ldots, n$,
%
%\ct{I think we need more conditions here on how a symbol $\con{f}_i$ can occur in $\varphi_j$.}
%\jb{do we? I thought that the admissibility criterion below is enough.}
% should be enough, unless I'm missing something? -ajr
and $\psi$ contains no function definitions.
We call $\psi$ the \emph{conjecture} of $\varphi$.
%\ct{Don't we need additional restrictions on the $\varphi_i$'s here?
%(E.g., they are quantifier-free, or perhaps existential or containing only let binders.)}
% Now fixed above -ajr
\end{definition}

In the signature $\Sigma$, we distinguish a subset $\sfundefs{\Sigma}
\subseteq \sfuns{\Sigma}$ of \emph{defined} uninterpreted function symbols.
We consider $\Sigma$-formulas that are in definitional form with respect to
$\sfundefs{\Sigma}$.

\newcommand\closurefmla{
  \ensuremath{\psi \models_T \bigwedge\nolimits_{\,i=1}^{n} \{ \varphi_i[ \vec t / \vec x ] \mid \con{f}_i( \vec t ) \in \terms( \psi ) \}} }

\begin{definition}\rm
Given a set of function definitions 
$\DDD = \{ \forallf{\con{f}_1} \vec x.\; \varphi_1, \ldots, \forallf{\con{f}_n} \vec x.\; \varphi_n \}$, 
a ground formula $\psi$ 
%\ct{I suppose $\psi$ is meant to be ground.} yes --ajr
is \emph{closed under function expansion with respect to $\DDD$} if
\begin{conf}$\closurefmla$. \end{conf}%
\begin{rep}\[\closurefmla\]\end{rep}%
The set $\DDD$ is \emph{admissible} if for every $T$-satisfiable formula 
$\psi$ closed under function expansion with respect to $\DDD$,
the formula
$\psi \wedge \bigwedge \Delta$ is also $T$-satisfiable.
%\ct{I}
\end{definition}

Admissibility is a semantic criterion that must be satisfied for each function
definition before applying the translation described in
Section~\ref{sec:encoding}. It is interesting to connect it to the
standard notion of \emph{well-founded} function definitions, often called
\emph{terminating} definitions in a slight abuse of terminology. In such
definitions, all recursive calls are decreasing with respect to a well-founded
relation, which must be supplied by the user or inferred automatically
using a termination prover. This ensures that the function is uniquely defined
at all points.

First-order logic has no built-in notion of computation or termination. To ensure
that a function specification is well-founded, it is sufficient to require that
the function would terminate when seen as a functional program, under \emph{some}
evaluation order. For example, the definition
$\forall x : {\typeint}.\;\, \lite\bigl(
x \leq 0,\allowbreak\; \con{p}( x ) \teq 1,\allowbreak\; \con{p}( x ) \teq 2 * \con{p}( x - 1 ) \bigr)$,
where $T$ is integer arithmetic extended with the uninterpreted symbol 
$\con p:\typeint \rightarrow \typeint$, can be shown well-founded under a strategy that
evaluates the condition of an $\lite$ before evaluating the relevant branch,
ignoring the other branch. \begin{rep}Logically, such dependencies can be captured by
congruence rules. \end{rep}Krauss developed these ideas in the more general context of
higher-order logic \cite[Section 2]{krauss-2009-phd}.

\begin{theorem}\label{thm:adm}
If $\DDD$ is a set of well-founded function definitions for\/
$\sfundefs{\Sigma} \begin{rep} = \{\con{f}_1,\ldots,\con{f}_n\}\end{rep}$, then it is admissible.
\end{theorem}
\begin{rep}
\begin{proofsketch}
Let $\psi$ be a satisfiable formula closed under function expansion with
respect to $\DDD$. We show that $\psi \wedge \bigwedge \Delta$ is also
satisfiable. Let $\I$ be a model of $\psi$, and
let $\I_0$ be the restriction of $\I$ to the function symbols in $\sfuns{\Sigma} -
\sfundefs{\Sigma}$. Because well-founded definitions uniquely characterize
the interpretation of the functions they define, there exists a
$\Sigma$-interpretation $\J$ that extends $\I_0$ such that $\J \models \Delta$.
%
Since $\psi$ is closed under function expansion, it already constrains the
functions in $\sfundefs{\Sigma}$ recursively as far as is
necessary for interpreting $\psi$. Thus, any point $v$ for which
$\const{f}_i^\I(v)$ is needed for interpreting $\psi$ will have its
expected value according to its definition and hence coincide with $\J$.
And since $\psi^\I$ does not depend on the interpretation at the other
points, $\J$ is, like $\I$, a model of $\psi$.
Since $\J \models \Delta$ by assumption, we have $\J \models \psi \wedge
\bigwedge \Delta$ as desired.
%We have
%$\varphi \,\models_T\, \bigwedge\nolimits_{\,i=1}^{n} \{ \varphi_i[ \vec t / \vec x ] \mid \con{f}_i( \vec t ) \in \terms( \varphi ) \}$.
\qed
\end{proofsketch}
\end{rep}

TODO: Say something about admissibility for tail-recursion: Watch out if the
argument always changes...

\begin{rep}
Another useful fragment of function definitions is the class of
\emph{productive} corecursive functions. Corecursive functions are functions to
a coalgebraic datatype. These functions can be non-well-founded, without
their being inconsistent. Productive corecursive functions are functions that
progressively reveal parts of their potentially infinite
output \cite{turner-1995,mcbride-productive}.
For example, given a type of infinite streams constructed by
$\con{scons} : \ty{int} \times \ty{stream} \to \ty{stream}$,
the definition
$\forallf{\const{e}} x.\;\, \const{e}(x) \teq \con{scons}(x,\; \const{e}(x + 1))$
falls within this fragment: Each call to $\const{e}$ produces one
constructor before entering the nested call. Like terminating recursion,
productive corecursion totally specify the functions it defines.
It is even possible to mix recursion and corecursion in the same function
\cite{blanchette-et-al-2015-fouco}. Theorem~\ref{thm:adm} can be extended to
cover such specifications, based on the observation that unfolding
a corecursive definition infinitely computes a unique infinite object.

Beyond totality, an admissible set can contain underspecified functions
such as $\forallf{\con{f}} x : \typeint.\allowbreak\;\, \con{f}( x )
\teq \con{f}( x - 1 )$ or $\forallf{\con{g}} x.\allowbreak\;\, \con{g}( x
) \teq \con{g}( x )$. We conjecture that one can ignore all
tail-recursive calls (i.e., calls that occupy the right-hand side of the
definition, potentially under some $\lite$ branch) when establishing well-foundedness
or productivity, without affecting admissibility.
\end{rep}

%\newcommand\badassex{
% \{\forallf{\con{f}} x : \typeint.\;\, \con{f}( x ) \teq \con{f}( x ),\;\,
% \forallf{\con{g}} x : \typeint.\;\, \lite( \con{f}( x ) \teq 5,\allowbreak\;
%   \con{g}( x ) \teq \con{f}( x ),\allowbreak\;
%   \con{g}( x ) \teq \con{g}( x ) + 1)\}}
\newcommand\badassex{
 \{\forallf{\con{f}} x : \typeint.\;\, \con{f}( x ) \teq \con{f}( x ),\allowbreak\;\,
 \forallf{\con{g}} x : \typeint.\;\, \con{g}( x ) \teq \con{g}( x ) + \nobreak\con{f}( x )\}}

An example of an inadmissible set is
$\{ \forallf{\con{f}} x : \typeint.\;\, \con{f}( x ) \teq \con{f}( x ) + 1 \}$,
where $T$ is integer arithmetic extended to a set of uninterpreted symbols 
$\{\con f, \con g :\typeint \rightarrow \typeint,\, \ldots \}$.
The reason is that the formula $\ltrue$ is (trivially) closed under function expansion with respect to this set,
and there is no model of $T$
satisfying $\con{f}$'s definition. A more subtle example is
\[\badassex\]
While this set has a model where $\con f$ and $\con g$ are interpreted as the
constant function $0$, it is not admissible since %the formula
$\con{f}( 0 ) \teq 1$ is
closed under function expansion \begin{rep}with respect to this set\end{rep}
and yet there exists
no interpretation satisfying both $\con{f}( 0 ) \teq 1$ and $\con{g}$'s
definition.

%Terminating function specifications are always admissible, but also some
%nonterminating
%An admissible function definition need not specify a terminating function, e.g.
%$\{ \forallf{\con{f}} x^\typeint.\;\, \con{f}( x ) \teq \con{f}( x - 1 ) \}$ or even $\{ \forallf{\con{f}} x.\;\, \con{f}( x ) \teq \con{f}( x ) \}$ are admissible.

\section{The Translation}
\label{sec:encoding}

%%% TYPESETTING: approximated, to get centering
\newcommand{\itemx}{\itemindent6.35em\item}

\begin{figure}[t]
\normalsize
\begin{enumerate}
%\begin{framed}
\itemx[\ ]
$\conv_0( t\typ{\tau},\, p )$ $=$
%\\[-.8\baselineskip]
 \begin{itemize}
   \itemx[] $\mathsf{if}$ $\tau = \Bool$ $\mathsf{and}$ $t = \boolop(t_1,\ldots,t_n)$ $\mathsf{then}$
    \begin{itemize}
      \itemx[] $\mathsf{let}$ $( t'_i,\, \chi_i ) = \conv_0( t_i,\, \pol( \boolop,\, i,\, p ) )$ $\mathsf{for}$ $i = 1, \ldots, n$ $\mathsf{in}$%\kern6.35em$\mid$
      \itemx[] $\mathsf{let}$ $\chi = \chi_1 \wedge \cdots \wedge \chi_n$ $\mathsf{in}$
      \itemx[] $\mathsf{if}$ $p = \ppos$ $\mathsf{then}$ \bigtuple{$\boolop(t'_1, \ldots, t'_n) ) \wedge \chi,\, \ltrue$}
      \itemx[] $\mathsf{else}$ $\mathsf{if}$ $p = \pneg$ $\mathsf{then}$ \bigtuple{$\boolop(t'_1, \ldots, t'_n) \vee \neg \chi,\, \ltrue$}
      \itemx[] $\mathsf{else}$ \bigtuple{$\boolop(t'_1, \ldots, t'_n),\, \chi$}
%\\[-.8\baselineskip]
    \end{itemize}
  \itemx[] $\mathsf{else}$ $\mathsf{if}$ $t = \forallf{\con{f}} \vec x.\;\, u$ $\mathsf{then}$
    \begin{itemize}
      \itemx[] $\mathsf{let}$ $( u',\, \chi ) = \conv_0( u,\, p )$ $\mathsf{in}$ \bigtuple{$\forall a : \fargtype{\con{f}}.\; u' [ \vecfarg{\con{f}}( a ) / \vec x ],\, \ltrue$}
%\\[-.8\baselineskip]
    \end{itemize}
  \itemx[] $\mathsf{else}$ $\mathsf{if}$ $t = \forall \vec x.\;\, u$ $\mathsf{then}$
    \begin{itemize}
      \itemx[] $\mathsf{let}$ $( u',\, \chi ) = \conv_0( u,\, p )$ $\mathsf{in}$ \bigtuple{$\forall \vec x.\; u',\, \forall \vec x.\; \chi$}
%\\[-.8\baselineskip]
    \end{itemize}
   \itemx[] $\mathsf{else}$
   \begin{itemize}
     \itemx[] \return \bigtuple{$t,\; \bigwedge\, \{ \exists a : \fargtype{\con{f}}.\; \vecfarg{\con{f}}( a ) \teq \vec s \mid \con{f}( \vec s ) \in \terms( t ),\, \con{f} \in \sfundefs{\Sigma} \}$}
%\\[-.8\baselineskip]
   \end{itemize}
 \end{itemize}
\end{enumerate}
\begin{enumerate}
%\begin{framed}
\itemx[\ ]
$\conv( \varphi )$ $=$ $\mathsf{let}$ $( \varphi',\, \chi ) = \conv_0( \varphi,\, \ppos )$ $\mathsf{in}$ $\varphi'$
%\\[-1.25\baselineskip] %% TYPESETTING
\end{enumerate}
\caption{\,Definition of translation $\conv$}
\label{fig:A}
\end{figure}

For the rest of the section, let $\varphi$ be a $\Sigma$-formula
in definitional form with respect to $\sfundefs{\Sigma}$
whose definitions are admissible.
%We may translate $\varphi$ into a equisatisfiable formula $\phi'$ for which known model-finding procedures~\cite{GeDeM-CAV-09, ReyEtAl-1-RR-13} are applicable.
We present a method that constructs an extended signature
$\extendsig{ \Sigma }$ and an $\extendsig{ \Sigma }$-formula $\varphi'$ such that
$\varphi'$ is $T$-satisfiable if and only if $\varphi$ is $T$-satisfiable---i.e.,
$\varphi$ and $\varphi'$ are \emph{equisatisfiable} (\emph{in $T$}).
The idea behind this translation
is to use an uninterpreted type to abstract the set of
\emph{relevant} tuples for each defined function $\con{f}$ and restrict the
quantification of $\con{f}$'s definition to a variable of this
type. 
Informally, the relevant tuples $\vec t$ of a function $\con{f}$ are the
ones for which the interpretation of $\con{f}( \vec t )$ is relevant to the
satisfiability of $\varphi$.
More precisely,
%suppose our signature $\Sigma$ contains a set of defined function symbols $\sfundefs{\Sigma} \subseteq \sfuns{\Sigma}$.
for each $\con{f} : \tau_1 \times \cdots \times \tau_n \rightarrow \tau \in \sfundefs{\Sigma}$,
the extended signature $\extendsig{\Sigma}$ contains
%\begin{itemize}
%\item
an uninterpreted \emph{abstract type} $\fargtype{\con{f}}$ and
%\item
$n$ uninterpreted \emph{concretization functions} $\fargx{\con{f}}{1} : \fargtype{\con{f}} \rightarrow \tau_1$, \ldots, $\fargx{\con{f}}{n} : \fargtype{\con{f}} \rightarrow \tau_n$.
%\end{itemize}
%The interpretation of uninterpreted type $\fargtype{\con{f}}$ will denote the elements (tuples) on which the function $\con{f}$ is applied.
%The role of the uninterpreted functions $\farg{\con{f}}^1$, $\ldots$, $\farg{\con{f}}^n$ will be discussed more in the following.

The translation $\conv$ defined in Figure~\ref{fig:A} translates the $\Sigma$-formula
$\varphi$ into the $\extendsig{\Sigma}$-formula $\varphi'$. It relies
on the auxiliary function $\conv_0$, which takes two arguments:\ the term $t$
to translate and a \relax{polarity}~$p$ for $t$, which is either $\ppos$, $\pneg$, or
$\pnone$. 
\vthinspace$\conv_0$ returns a pair $( t'\negvthinspace,\, \chi )$, where $t'$ is a term of
the same type as $t$ and $\chi$ is an $\extendsig{\Sigma}$-formula.
%an additional constraint.
%This constraint is needed if the polarity is $\pnone$; otherwise, it is encoded
%directly in $t'$ and $\chi$ is simply $\ltrue$.
%In the auxiliary function $\conv_0$, the function $\pol(
%\boolop, i, p )$ returns the polarity of the $i$\vvthinspace{th} argument of an
%application of~$\boolop$ having polarity $p$ if $\boolop$ is an interpreted
%predicate or $\pnone$ otherwise.


%The role of $D$ is to ensure that
%the (restricted) function definition for $\con{f}$ includes certain tuples in
%its domain, which we explain more in the following.

The translation alters the formula $\varphi$ in two ways. First, it restricts the
quantification on function definitions for $\con{f}$ to the corresponding
uninterpreted type $\fargtype{\con{f}}$, inserting applications of the concretization functions $\fargx{\con{f}}{i}$ as needed. 
Second, it augments $\varphi$ with additional constraints of the form
$\exists a :\nobreak \fargtype{\con{f}}.\;\, \vecfarg{\con{f}}( a ) \teq \vec s$,
where $\vecfarg{\con{f}}( a ) \teq \vec s$ abbreviates the formula
$\bigwedge_{i=1}^n\, \fargx{\con{f}}{i}(a) \teq s_i$
with $\vec s = (s_1,\ldots,s_n)$.
These existential %%% for typesetting
constraints
ensure that the restricted definition for $\con{f}$ covers all relevant tuples
of terms, namely those occurring in applications of $\con{f}$ 
that are relevant to the satisfiability of $\varphi$.

If $t$ is an application of a predicate symbol $\boolop$, including the
operators ${\oldneg}$, ${\wedge}$, ${\vee}$, ${\teq}$, and ${\lite}$,
$\conv_0$ calls itself recursively on the arguments $t_i$ and polarity $\pol(
\boolop, i, p )$, with $\pol$ defined as
\[\pol( \boolop, i, p ) =
\begin{cases}
p & \text{if either $\boolop \in \{{\wedge}, {\vee}\}$ or $\boolop = \lite$ and $i \in \{2, 3\}$} \\[-\jot]
\pnegate p & \text{if $\boolop = {\neg}$} \\[-\jot]
\pnone & \text{otherwise}
\end{cases}\]
where $\pnegate p$ is $\pneg$ if $p$ is $\ppos$, $\ppos$
if $p$ is $\pneg$, and $\pnone$ if $p$ is $\pnone$.
The term $t$ is then reconstructed as $\boolop(t_1',\ldots,t_n')$ 
where each $t_i'$ is the result of the recursive call with argument $t_i$. 
If the polarity $p$ associated with $t$ is $\ppos$, 
$\conv_0$ conjunctively adds to $\boolop(t_1',\ldots,t_n')$ the constraint $\chi$ derived from
the subterms. Dually, if $p$ is $\pneg$, it adds a
disjunction with the negated constraint, to achieve the same overall effect.
%ct  as in the positive case. 
It $p$ is $\pnone$, the constraint $\chi$ is
returned to the caller.
%
\begin{rep}\par\end{rep}
%
If $t$ is a function definition, % of a function in $\sfundefs{\Sigma}$,
$\conv_0$ constructs a quantified formula over a single variable
$a$ of type $\fargtype{\con{f}}$ and replaces all occurrences of $\vec x$ in
the body of that formula with $\vecfarg{\con{f}}( a )$. 
(Since function definitions are top-level conjuncts, %by case analysis on the return values of $\conv$,
$\chi$ must be $\ltrue$ and can be ignored.)
%
If $t$ is an unannotated quantified formula, $\conv_0$ calls itself
on the body with the same polarity. \begin{rep}A quantifier is prefixed to the
quantified formula and constraint returned by the recursive call.\end{rep}
Otherwise, $t$ is either an application of an uninterpreted predicate symbol or a term
of a type other than $\Bool$. Then, the returned constraint is a conjunction of
formulas of the form $\exists a : {\fargtype{\con{f}}}.\;\, \vecfarg{\con{f}}(
a ) \teq \vec s$ for each subterm $\con{f}( \vec s )$ of~$t$ such that $\con{f}
\in \sfundefs{\Sigma}$. Such constraints, when asserted positively, ensure that
some element in the abstract domain $\fargtype{\con f}$ is the preimage of
the argument tuple $\vec s$. % under $\vecfarg{\con{f}}$.

%We demonstrate this translation with an example.

\begin{example}
\label{ex:translation}
Let $T$ be linear arithmetic with the uninterpreted symbols  $\{ \const{c} : { \typeint },\> \con{s} : { \typeint \rightarrow \typeint } \}$.
Let $\varphi$ be the $\Sigma$-formula
\begin{equation} \label{eq:ex-before}
\forallf{\con{s}} x : {\typeint}.\;\, \lite\bigl( x \leq 0,\; \con{s}(x) \teq 0,\;
  \con{s}( x ) \teq x + \con{s}( x - 1 ) \bigr) \wedge \con{s}( \con{c} ) > 100
\end{equation}
%
The definition of $\const{s}$ specifies that it returns the sum of all
positive integers up to $x$. The formula $\varphi$ is in definitional form with
respect to $\sfundefs{\Sigma}$ 
%\jb{correct? It said $\Sigma_u$, which was definitely wrong} 
and states that the sum of all
positive numbers up to some constant $\const{c}$ is greater than $100$. It is
satisfiable with a model that interprets $\const{c}$ as $14$ or more.
Due to the universal quantifier,
current SMT techniques
are unable to find
a model for $\varphi$. The signature $\extendsig{\Sigma}$ extends $\Sigma$ with the type
$\fargtype{\con{s}}$ and the uninterpreted function symbol $\farg{\con{s}} : \fargtype{\con{s}}
\rightarrow \typeint$. The result of $\conv( \varphi )$, after simplification, 
is the $\extendsig{\Sigma}$-formula
%
\begin{equation} \label{eq:ex-after}
\!\begin{aligned}[c]
  & \phantom{{\wedge}\; }\bigl(
      \forall a : \fargtype{\con{s}}.\; \lite\bigl(
        \!\begin{aligned}[t]
         &  \farg{\con{s}}( a ) \leq 0,\;
          \con{s}(\farg{\con{s}}( a )) \teq 0,\;
\\[-\jot]
  & \con{s}(\farg{\con{s}}( a )) \teq \farg{\con{s}}( a ) + \con{s}( \farg{\con{s}}( a )-1 )
    \wedge \exists b : {\fargtype{\con{s}}}.\;\, \farg{\con{s}}( b ) \teq \farg{\con{s}}( a )-1 \bigr) \bigr)
\end{aligned}
\\[-\jot]
 & {\wedge}\; \con{s}( \con{c} ) > 100 \wedge \exists a : {\fargtype{\con{s}}}.\;\, \farg{\con{s}}( a ) \teq \con{c}
\end{aligned}
\end{equation}
%
The universal quantifier in formula~(\ref{eq:ex-after}) ranges over an uninterpreted
type $\fargtype{\con{s}}$, making it amenable to the finite model finding
techniques by Reynolds et al.\ \cite{ReyEtAl-1-RR-13,reynolds-et-al-2013},
implemented in \cvc, which search for a finite interpretation for $\fargtype{\con{s}}$. 
Furthermore, since all occurrences of the quantified variable~$a$ are 
beneath applications of the uninterpreted function $\farg{\con{s}}$, 
the formula is in the essentially uninterpreted fragment,
for which Ge and de Moura \cite{GeDeM-CAV-09} provide 
a complete instantiation procedure, implemented in \ziii. 
As expected,
%%% Are we sure?  
% Yes, I'm planning to post the 2 versions of the benchmark on the evaluation page (when we add it) --ajr
% Yes, but how can you be sure that Z3 will run forever? Your benchmarks surely have a timeout! --jb
\cvc and \ziii run indefinitely on formula~(\ref{eq:ex-before}), 
whereas they produce a model for~(\ref{eq:ex-after}) 
within 100 milliseconds.\xend
\end{example}

\newcommand\badlambda{\lambda x : \typeint.\allowbreak\;\, \lite( x \teq 0,\allowbreak\; 0,\allowbreak\;
  \lite( x \teq 1,\allowbreak\; 1,\allowbreak\;
    \lite( x \teq 2,\allowbreak\; 3,\allowbreak\;
      \lite(\ldots,\allowbreak\; \lite( x \teq 13,\allowbreak\, 91,\allowbreak\, 105 )\negvthinspace \ldots ))))}

Note that the translation $\conv$ results in formulas whose models
(i.e., satisfying interpretations) are generally different from those of $\varphi$.
%\ct{"Preserves models" means that it does not drop models, not that it adds models.}
One model $\I$ for formula~(\ref{eq:ex-after}) in the above example interprets
$\fargtype{\con{s}}$ as a finite set $\{ u_0, \ldots, u_{14} \}$,
$\farg{\con{s}}$ as a finite map $u_i \mapsto i$ for $i = 0, \ldots, 14$,
$\con{c}$ as $14$,
and $\con{s}$ as the almost constant function
%
%\begin{equation} \label{eq:approx-interp}
\begin{conf}$\badlambda$. \end{conf}%
\begin{rep}\[\badlambda\]\end{rep}%
%\end{equation}
%
In other words, $\const{s}$ is interpreted as a function mapping $x$ to the sum
of all positive integers up to $x$ when $0 \leq x \leq 13$, and $105$
otherwise.
The $\Sigma$-reduct of $\I$ is not a model of the original formula~(\ref{eq:ex-before}),
since $\I$ 
%ct wrongly in what sense?
%wrongly 
interprets $\con{s}( n )$ as $105$ when $n < 0$ or $n > 14$.

However, under the assumption that the function definitions in
$\sfundefs{\Sigma}$ are admissible, %we claim that
$\conv(\varphi)$ is equisatisfiable with $\varphi$ for any input $\varphi$.
%\footnote{
%ct moved in a footnote to avoid breaking the flow
%jb killed altogether
%For the previous example, the intuition is that the
%interpretation of a term such as $\con{s}( -1 )$ is not
%relevant to the satisfiability of formula~(\ref{eq:ex-before}). 
%}
%ct removed because obvious
%Thus, a ``satisfiable'' or ``unsatisfiable'' response from an SMT solver on input $\conv(
%\varphi )$ implies the existence or nonexistence of a model for
%$\varphi$. 
Moreover, the models of $\conv( \varphi )$ contain
pertinent information about the models of $\varphi$. For example, the model
$\I$ for formula~(\ref{eq:ex-after}) given above interprets $\con{c}$ as $14$
and $\con{s}(n)$ as $\sum_{i=1}^n i$ for $0 \le n \le 14$,
and there exists a model of formula~(\ref{eq:ex-before}) that also interprets
$\con{c}$ and $\con{s}(n)$ in the same way (for $0 \le n \le 14$).
In general, for every model of $\conv( \varphi )$,
there exists a model of $\varphi$ that
coincides with it on its interpretation of all
function symbols in $\sfuns{\Sigma} - \sfundefs{\Sigma}$.
%($\{\con{s} \}$ in the example
%\ct{This is not right. To start, the set difference should contain all the arithmetic symbols here. Moreover, isn't $\con s$ in $\sfundefs{\Sigma}$?})
Furthermore, the model of $\conv( \varphi )$ will
also give correct information for the defined functions at all points belonging
to the domains of the corresponding abstract types. This can sometimes help
users debug their specifications.

%%% @ANDY: I don't get the following comment --jb
%   @JASMIN : The point of this comment is that the user is not interested in the SMT solver inferring any new information regarding the interpretation of s,
%             since s is a "defined" function, the user already has an intended interpretation in their head of what s is (here, the sum of positive integers between 0...x).
%             Thus, it isn't a big deal that the SMT solver "approximates" the interpretation of s during a check, since we already know what s is.
%From a practical point of view, this is not an
%issue, because $s$ is the very function that was explicitly defined by the
%user, and hence already has an intended interpretation.

%%% @ANDY: Now I understand. My experience with Nitpick is that users sometimes
%%% want to see the partial information available about defined functions. I've
%%% added a couple of sentences above to that effect.

\begin{rep}
%In the following,
We sketch the correctness of translation~$\conv$.
For a set of ground literals~$L$, 
we write $\absconstraints( L )$ to denote the set of constraints that
force the concretization functions to have the necessary elements in their
range for determining the satisfiability of $L$ with respect to the function
definitions in the translation.
Formally, we define
$\absconstraints( L )$
as
$\{ \exists a : {\fargtype{\con{f}}}.\;\, \vecfarg{\con{f}}( a ) \teq \vec t
\mid \con{f}( \vec t ) \in \terms( L ),\: \con{f} \in \sfundefs{\Sigma}
\}$.
The following lemma states the central invariant behind the translation $\conv$.

\begin{lemma}\label{lem:conv}
Let $\psi$ be a formula not containing function definitions,
and let $\I$ be an $\extendsig{\Sigma}$-interpretation.
Then $\I$ satisfies $\conv( \psi )$ if and only if
$\I$ satisfies $L \cup \absconstraints( L )$, where $L$ is a set of ground $\Sigma$-literals that entail $\psi$.
%There exists a set of ground $\Sigma$-literals $L$ such that $L$ entails $\varphi$, 
%and $\I$ satisfies $L \cup \absconstraints( L )$.
\end{lemma}
\begin{proofsketch}
By definition of $\conv$ and case analysis on the return values of $\conv_0$.
\qed
\end{proofsketch}

\begin{corollary}\label{cor:conv}
If $\psi$ is a formula not containing function definitions, then $\conv( \psi )$ entails $\psi$.
\end{corollary}
\end{rep}

\begin{theorem}\label{thm:equi}
If $\varphi$ is a\/ $\Sigma$-formula in definitional form with respect to\/
$\sfundefs{\Sigma}$
and the set of function definitions\/ $\DDD$ corresponding to\/
$\sfundefs{\Sigma}$ is admissible, then\/ $\varphi$ and\/ $\conv( \varphi )$ are
equisatisfiable in $T$.
\end{theorem}
\begin{rep}
\begin{proofsketch}
First, we show that if $\varphi$ is satisfied by an
$\Sigma$-interpretation $\I$, then $\conv( \varphi )$ is satisfied by an $\extendsig{\Sigma}$-interpretation $\J$.
Let $\J$ be the $\extendsig{\Sigma}$-interpretation that interprets all types $\tau \in \stypes{\Sigma}$ as~$\tau^\I$,
all functions $\con{f} \in \sfuns{\Sigma}$ as $\con{f}^\I$,
and for each function $\con{f} : \tau_1 \times \cdots \times \tau_n \rightarrow \tau$ in $\sfundefs{\Sigma}$,
interprets $\fargtype{\con{f}}$ as $\tau_1^\I \times \cdots \times \tau_n^\I$
and each $\fargx{\con{f}}{i}$ as the $i$\vvthinspace{th} projection on such tuples for $i = 1, \ldots, n$.
Since $\J$ satisfies $\varphi$, it satisfies a set of ground literals $L$ that entail $\varphi$.
Furthermore, $\J$ satisfies \relax{every} constraint of the form $\exists a : {\fargtype{\con{f}}}.\;\, \vecfarg{\con{f}}( a ) \teq \vec t$,
since by our construction of $\J$ there exists a value $v \in {\fargtype{\con{f}}}^\J$ such that $v = \vec t^\J$.
%By case analysis it follows that $\conv_0$ maintains the invariant that
%if $\conv_0( t\typ{\Bool}, \ppos )$ returns $( {t'}, \chi )$,
%then $\I$ satisfies $t$ if and only if $\J$ satisfies $t'$.
Thus, $\J$ satisfies $L \cup \absconstraints( L )$, and by Lemma~\ref{lem:conv}
we conclude $\J$ satisfies $\conv( \varphi )$.

Second, we show that if $\conv( \varphi )$ is satisfied by a
$\extendsig{\Sigma}$-interpretation $\J$, then $\varphi$ is satisfied by a
$\Sigma$-interpretation $\I$. Since $\varphi$ is in definitional form with
respect to the functions defined by $\DDD$, it must be of the form
$\DDD \wedge \varphi_0$. 
%We argue that $\J$ satisfies a $\Sigma$-formula
%$\varphi'$ that is closed under function expansion with respect to $\DDD$, and
%where $\varphi'$ entails $\varphi_0$.
%To construct $\varphi'$, 
First, we define a sequence of $\Sigma$-literals sets $L_0 \subseteq L_1 \subseteq \cdots$
such that $\J$ satisfies $L_i \cup \absconstraints( L_i )$ for $i = 0, 1, \ldots\vthinspace$.
Since $\J$ satisfies $\conv( \varphi_0 )$,
by Lemma~\ref{lem:conv}, 
$\J$ satisfies a set of literals $L \cup \absconstraints( L )$ where $L$ is a set of $\Sigma$-literals that entail $\varphi_0$.
Let $L_0 = L$.
For each $i \geq 0$,
let $\psi_i$ be the formula $\bigwedge\, \{ \conv( \varphi_{\con{f}}[ \vec t / \vec x ] ) \mid \con{f}( \vec t ) \in \terms( L_i ),\: \con{f} \in \sfundefs{\Sigma} \}$,
where $\forallf{\con{f}} \vec x.\; \varphi_{\con{f}} \in \DDD$.
Since $\J$ satisfies $\conv( \forallf{\con{f}} \vec x.\; \varphi_{\con{f}} )$ and $\absconstraints( L_i )$,
we know that $\J$ also satisfies $\psi_i$.
Thus by Lemma~\ref{lem:conv},
$\J$ satisfies a set of literals $L \cup \absconstraints( L )$ where $L$ is a set of $\Sigma$-literals that entail $\psi_i$.
Let $L_{i+1} = L_0 \cup L$.
Let $L_\infty$ be the limit of this sequence (i.e., $\ell \in L_\infty$ if and only if
$\ell \in L_i$ for some $i$),
and let $\psi$ be the $\Sigma$-formula $\bigwedge L_\infty$.
To show that $\psi$ is closed under function expansion with respect to $\DDD$,
we first note that by construction $\psi$ entails $\psi_\infty$.
For any function symbol~$\con{f}$ and terms~$\vec t$, since $\varphi_{\con{f}}[ \vec t / \vec x ]$ does not contain function definitions,
by Corollary~\ref{cor:conv},
$\conv( \varphi_{\con{f}}[ \vec t / \vec x ] )$ entails $\varphi_{\con{f}}[ \vec t / \vec x ]$.
Thus, $\psi$ entails $\{ \varphi_{\con{f}}[ \vec t / \vec x ] \mid \con{f}( \vec t ) \in \terms( \psi ), \con{f} \in \sfundefs{\Sigma} \}$,
meaning that $\psi$ is closed under function expansion with respect to $\DDD$.
Furthermore, $\psi$ entails $\varphi_0$ since $L_0 \subseteq L_\infty$.
Since $\psi$ is a $T$-satisfiable formula that is closed under function expansion with respect to $\DDD$ and $\DDD$ is admissible, 
by definition there exists a $\Sigma$-interpretation $\I$ satisfying $\psi \wedge \DDD$, which entails $\DDD \wedge \varphi_0$,
i.e., $\varphi$.
\qed
\end{proofsketch}

The intuition of the above proof is as follows.
First, $\conv( \varphi )$ cannot be unsatisfiable when $\varphi$ is satisfiable
since any $\Sigma$-interpretation that satisfies $\varphi$ can be extended in a straightforward way to 
an $\extendsig{\Sigma}$-interpretation that satisfies $\conv( \varphi )$.
Second, if a model is found for $\conv( \varphi )$,
then the constraints $\exists a : \fargtype{\con{f}}.\;\, \vecfarg{\con{f}}( a ) \teq \vec t$
occurring in $\conv( \varphi )$ ensure that this model also satisfies
a $\Sigma$-formula that is closed under function expansion and
that entails the conjecture of $\varphi$.
This implies the existence of a model for $\varphi$,
assuming $\DDD$ is admissible.
%For clarity, we demonstrate the consequences of this theorem by means of an example.
\end{rep}

\begin{conf}
We give an intuition of the above theorem in the context of an example.
For a set of ground $\Sigma$-literals $L$, let $\absconstraints( L )$ be the set
$\{ \exists a : {\fargtype{\con{f}}}.\;\, \vecfarg{\con{f}}( a ) \teq \vec s
\mid \con{f}( \vec s ) \in \terms( L ),\: \con{f} \in \sfundefs{\Sigma}
\}$.
\end{conf}

\begin{example}
Let us revisit the formulas in Example~\ref{ex:translation}. 
%ct rephrased in more direct terms.
%This formula is unsatisfiable only if
%formula~(\ref{eq:ex-before}) is unsatisfiable, since $\fargtype{\con{s}}$ can be
%interpreted as the integers and $\farg{\con{s}}$ as the identity
%function. 
If the original formula~(\ref{eq:ex-before}) is $T$-satisfiable, the translated
formula~(\ref{eq:ex-after}) is clearly also $T$-satisfiable since $\fargtype{\con{s}}$ can be
interpreted as the integers and $\farg{\con{s}}$ as the identity function. 
%
Conversely, we claim that~(\ref{eq:ex-after}) is $T$-satisfiable
only if~(\ref{eq:ex-before}) is $T$-satisfiable, noting that the set
$\{ \forallf{\con{s}} x.\;\, \varphi_\con{s} \}$ is admissible,
where $\varphi_\con{s}$ is the formula $\lite\bigl( x \leq 0,\allowbreak\; \con{s}(x) \teq\nobreak 0,\allowbreak\; \con{s}( x ) \teq x + \con{s}( x-1 ) \bigr)$.
%\ct{
%You cannot possibly mean "if and only if" in the two sentences above.
%Once I show that (1) is unsat iff (2) is unsat, I do not need to argue that
%(1) is sat iff (2) is sat. That is immediate.
%}
% You're right, changed both to "only if" --ajr
%As stated
%in the proof of Theorem~\ref{thm:equi}, the reason is that any model of
%formula~(\ref{eq:ex-after}) must satisfy a formula that is closed under
%function expansion, hence implying that formula~(\ref{eq:ex-before}) has a
%model.
%The rightmost conjunct $\exists a : {\fargtype{\con{s}}}. (\farg{\con{s}}( a ) \teq \con{c})$ 
%ensures that if $\con{s}( \con{c} ) > 100$ is a constraint that 
%formula~(\ref{eq:ex-after}) satisfying $\con{s}( \con{c} ) > 100$ also
%satisfies $\varphi_{\con{s}}[\con{c}/x]$. 
%The nested existential
%$\exists b.\;\, \farg{\con{s}}( b ) \teq \farg{\con{s}}( a )-1$ ensures that
%a similar constraint is enforced recursively: when $\varphi_{\con{s}}[t/x]$ is used in part to evaluate $\con{s}( \con{c} )$,
%any model of $\varphi_{\con{s}}[t/x]$ must also satisfy $\varphi_{\con{s}}[t-1/x]$ when $t > 0$. 
Clearly, any interpretation $\I$ satisfying formula~(\ref{eq:ex-after}) satisfies 
$L_0 \cup \absconstraints( L_0 )$,
where $L_0 = \{ \con{s}( \con{c} ) > 100 \}$
and $\absconstraints( L_0 )$ consists of the single constraint
$\exists a : {\fargtype{\con{s}}}.\;\, \farg{\con{s}}( a ) \teq \con{c}$.
%The latter ensures that $\I$ satisfies a formula that entails both $\con{s}( \con{c} ) > 100$ and $\varphi_{\con{s}}[\con{c}/x]$.
Since $\I$ also satisfies both the translated function definition for $\con{s}$
(the first conjunct of~(\ref{eq:ex-after})) and $\absconstraints( L_0
)$, it must also satisfy\begin{rep} the formula $\psi_1$:\end{rep}
\[
\lite\bigl( \con{c} \leq 0,\; 
            \con{s}(\con{c}) \teq 0,\;
            \con{s}(\con{c}) \teq \con{c} + \con{s}( \con{c}-1 )
            \wedge \exists b : {\fargtype{\con{s}}}.\;\, \farg{\con{s}}( b ) \teq \con{c}-1 \bigr)
\]
The existential constraint in the above formula
ensures that whenever $\I$ satisfies the set $L_1 = L_0 \cup \{ \neg \con{c} \leq 0,\; \con{s}(\con{c}) \teq \con{c} + \con{s}( \con{c}-1 ) \}$,
$\I$ satisfies $\absconstraints( L_1 )$ as well.
Hence, by repeated application of this reasoning, it follows that
a model of formula~(\ref{eq:ex-after}) that interprets $\con{c}$ as $n$ must also satisfy% the formula
~$\psi$:
%
\[\con{s}( \con{c} ) > 100 \wedge %\displaystyle
  \smash{\bigwedge\nolimits_{\vthinspace i=0}^{\!n-1}} \bigl( \neg (\con{c}-i \leq 0) \wedge \con{s}( \con{c}-i ) \teq \con{c}-i + \con{s}( \con{c}-i-1 ) \bigr)
\wedge \con{c}-n \leq 0 \wedge \con{s}( \con{c}-n ) \teq 0\]
%
This formula is closed under function expansion
since it entails $\varphi_\con{s}[(\con{c}-i)/x]$ for $i = 0, \ldots, n$,
and it contains only $\con{s}$ applications corresponding to $\con{s}( \con{c}-i )$ for $i = 0, \ldots, n$.
Since $\{\forallf{\con{s}} x.\;\, \varphi_\con{s}\}$ is admissible,
there exists a $\Sigma$-interpretation satisfying $\psi \wedge \forallf{\con{s}} x.\;\, \varphi_\con{s}$,
which entails formula~(\ref{eq:ex-before}).\xend
\end{example}

\section{Implementations}
\label{sec:implementations}

We have implemented the translation $\conv$ in two separate systems, 
as a preprocessor in the development version of \cvc (version~1.5 prerelease),
and in the higher-order model finder Nunchaku.
This section describes how the translation is implemented in both these systems,
as well as optimizations used by \cvc for finding models for the translated benchmarks.

%The translation $\conv$ significantly improves the effectiveness of current SMT
%techniques for finding models of formulas involving recursive
%function definitions.
%We have implemented it in the development version of \cvc %~\cite{ReyEtAl-1-RR-13}
%(version~1.5 prerelease).

\subsection{CVC4}
\label{ssec:cvc4}

In \cvc, function definitions $\forallf{\con{f}} \vec x.\; \varphi$ can be written using the $\definefunreccmd$ command
from the SMT-LIB 2.5 standard~\cite{smtlib25}.
Formula~(\ref{eq:ex-before}) from Example~\ref{ex:translation}
can be specified as %follows:
%
\begin{quote}
%\small -- no small, unless you convince me that these mixtures of font sizes
% are aesthetic and serve a purpose --jb
\begin{verbatim}
(define-fun-rec s ((x Int)) Int (ite (<= x 0) 0 (+ x (s (- x 1)))))
(declare-fun c () Int)
(assert (> (s c) 100))
(check-sat)
\end{verbatim}
\end{quote}
%
When reading this input,
\begin{rep}%
\cvc adds the annotated quantified formula
$$\forallf{\con{s}} x.\;\, \con{s}( x ) \teq \lite\bigl( x \leq 0,\; 0,\; \con{s}( x-1 )\bigr)$$
to its list of assertions,
which after rewriting becomes
$$\forallf{\con{s}} x.\;\, \lite\bigl( x \leq 0,\; \con{s}( x ) \teq 0,\; \con{s}( x ) \teq \con{s}( x-1 ) \bigr)$$
\end{rep}%
\begin{conf}%
\cvc adds
$\forallf{\con{s}} x.\;\, \con{s}( x ) \teq \lite\bigl( x \leq 0,\; 0,\; \con{s}( x-1 )\bigr)$
to its list of assertions,
which after rewriting becomes
$\forallf{\con{s}} x.\;\, \lite\bigl( x \leq 0,\; \con{s}( x ) \teq 0,\; \con{s}( x ) \teq \con{s}( x-1 ) \bigr)$. \end{conf}%
If \cvc's finite model finding mode for recursive functions is enabled (using
the command-line option \texttt{-}\texttt{-fmf-fun}), it will replace its list of known
assertions based on the translation $\conv$ before checking for satisfiability.
%If the functions provided in the input are admissible,
%then a ``satisfiable" response from the solver implies that a model exists for the original formula.
Accordingly, the solver will output the approximation of the interpretation it
used for recursive function definitions. 
For the example above, it outputs a %partial
model of~$\con{s}$ where only
the values of $\con{s}(x)$ for $x = 0,\ldots,14$ are correctly given:
\begin{quote}
%\small
\begin{verbatim}
(model
  (define-fun s (($x1 Int)) Int 
    (ite (= $x1 14) 105 (ite (= $x1 13) 91 (ite (= $x1 12) 78 
      (ite (= $x1 11) 66 (ite (= $x1 10) 55 (ite (= $x1 4) 10 
        (ite (= $x1 9) 45 (ite (= $x1 8) 36 (ite (= $x1 7) 28 
          (ite (= $x1 6) 21 (ite (= $x1 3) 6 (ite (= $x1 5) 15 
            (ite (= $x1 2) 3 (ite (= $x1 1) 1 0)))))))))))))))
  (define-fun c () Int 14))
\end{verbatim}
\end{quote}

The \texttt{-}\texttt{-fmf-fun} option tells \cvc to assume that
functions introduced using \texttt{define-\allowbreak fun-\allowbreak rec} are admissible.
We stress that admissibility must be discharged separately by the user---e.g., using a
syntactic criterion or a termination prover.
If some function definitions are not admissible, \cvc may answer \emph{sat} for
an unsatisfiable problem. 
\begin{rep}
Indeed, if we add the inconsistent definition
%
\begin{quote}
%\small
\begin{verbatim}
(define-fun-rec h ((x Int)) Int (+ (h x) x))
\end{verbatim}
\end{quote}
%
to the above problem and run \cvc with the \texttt{-}\texttt{-fmf-fun} option,
it wrongly answers \emph{sat}.
\end{rep}

\paragraph{Optimizations for Finding Finite Models of $\conv(\varphi)$}
Like other systems,
the finite model finding capability of \cvc incrementally fixes bounds on the cardinalities of uninterpreted types,
and increases these bounds until it encounters a model.
When multiple sorts are present, it uses a fairness scheme which
bounds the sum of cardinalities of all uninterpreted types~\cite{reynolds2013finite}.
For instance, if a signature has two uninterpreted types $\tau_1$ and $\tau_2$,
it will first search for models where 
the cardinality of $\tau_1$ plus the cardinality of $\tau_2$ is at most $2$,
then $3$, $4$, and so on.
To accelerate the search for models,
we use an optimization based on statically inferring \emph{monotone} types.
A monotone type is one in which models can always be extended with an additional element of that type.
Inference of monotone types has been show in previous work~\cite{claessen2011mono,korovin-2013}
to be an important component of finite model finding systems.
Moreover, it is easy to see that types $\fargtype{\con{f}}$ introduced by our translation $\conv$ are monotone.
The finite model finding capability of \cvc may take advantage of this fact by fixing the bounds for all 
monotone types simultaneously.
That is, if $\tau_1$ and $\tau_2$ are both monotone types (either introduced by our translation, or are otherwise inferred to be),
then we fix the bound for \emph{both} of these types to be $1$, then $2$, and so on.
This scheme allows the solver greater flexibility with respect to the default scheme,
and comes with no loss of generality with respect to models, since monotone sorts can always be extended to have equal cardinalities.

By default, 
\cvc uses techniques to minimize the number of literals it considers when constructing propositional satisfying assignments for formulas~\cite{relevancy2007}.
However, we have found such techniques degrade performance for finite model finding on
benchmarks having recursive functions that are defined by cases.
For this reason, we prefer disabling such techniques for benchmarks coming from our translation.

\ajr{More?}

%Usually not interested in showing logical consistency, there
%exist other tools that check e.g. termination.
%Thus, want SMT solver to check satisfiability the assumption that recursive
%functions are consistent (where this needs to be defined precisely).

\subsection{Nunchaku}
\label{ssec:nunchaku}

Nunchaku is a new model finder, intended
to be used for finding counter-examples in several proof assistants. The
first version, 0.1, was released in January 2016 with support for (co)datatypes,
(co)recursive functions and (co)inductive predicates.
Support for higher-order
functions is planned in the next release.
We intend for Nunchaku to be supported at least in Isabelle (where basic
support for it already exists), Coq, and the TLA\textsuperscript{+} proof
system.

Nunchaku is the spiritual successor to Nitpick~\cite{blanchette-nipkow-2010},
but is developed as a standalone OCaml program (with its own input language) so
as to be usable independently from Isabelle.
Whereas Nitpick relies on propositional checking with Kodkod,
generating a succession of problems where cardinalities of finite types
grow at each step, Nunchaku
translates its input to one first-order logic formula that targets
the finite-model finding fragment of CVC4, including the (co)datatypes support.
Using CVC4 will also allow Nunchaku to provide efficient arithmetic and makes
equality reasoning straightforward.

The following problem is expressed in Nunchaku's input syntax; it starts with
a datatype declaration, then two mutually recursive coinductive predicates
are defined, and finally a goal (``find a non-zero natural number $m$ that
is even'').
\begin{quote}
\begin{verbatim}
data nat := zero | Suc nat.

copred even : nat -> prop :=
  even zero;
  forall (n : nat). odd n => even (Suc n)
and odd : nat -> prop :=
  forall (n : nat). even n => odd (Suc n).

goal exists m. (even m && ~ (m = zero)).
\end{verbatim}
\end{quote}

Nunchaku parses and types the input problem before applying a sequence
of translations, each reducing the distance to the target fragment. Here,
for instance, the coinductive predicates
\verb!even! and \verb!odd! are \emph{polarized} (specialized into
\verb!even+!, used only in positive positions, and \verb!even-! used in negative
positions),
$m$ is Skolemized into a fresh constants,
then then translated into admissible functions
before another pass uses the encoding from Section~\ref{sec:encoding}.
If a model is found, it is translated back to the input language.
Conceptually, the sequence of transformation is a \emph{two ways pipeline}
built by composing pairs \textsf{(encode,decode)}.
For each such pair, \textsf{encode}
transforms a problem over a signature $\Sigma$ in a logic
$\mathcal{L}$ to a problem over a signature $\Sigma'$ in a slightly different
logic $\mathcal{L}'$, and \textsf{decode} translates a model of $\mathcal{L'}$
over $\Sigma'$ into a model of $\mathcal{L}$ over $\Sigma$. The pipeline
currently implemented in Nunchaku is described below.

\begin{description}
%{{{
  \item[Type inference] infers types and checks definitions;
  \item[Type skolemization]
    replaces $\exists \alpha. p[\alpha]$
      with $p[\tau]$ where $\tau$ is a fresh type;
  \item[Monomorphization]
    specializes polymorphic functions on their type arguments;
  \item[Elimination of Equations]
    translates multiple-equations definitions of functions into
      a single nested pattern matching;
  \item[Polarization]
    specializes some predicates into a version used in positive position
      and one used in negative position;
  \item[Unrolling]
    adds a decreasing argument to non-well-founded predicates;
  \item[Skolemization]
    introduces Skolem symbols for term variables;
  \item[Elimination of Predicates]
    changes a multi-clauses (co)inductive predicate definition
      into a single recursive function;
  \item[Recursion Elimination]
    performs the encoding from Section~\ref{sec:encoding};
  \item[Elimination of Match]
    transforms pattern-matching expressions into tests and selectors;
  \item[CVC4 Invocation] calls CVC4 to obtain a model.
%}}}
\end{description}

TODO: similar to institutions \cite{goguen-meseguer} (?)

\subsection{Decoding of Recursive Functions}

Nunchaku is designed for finding counter-models to formulas a user thinks
are theorems. Decoding models from CVC4 to hide any trace of the successive
encodings is therefore vital for readability. Each step
of the pipeline only has to concern itself with removing its changes to
the signature and logic; our encoding of recursive functions is no exceptions.

TODO explain how:
- domains of proj. functions are computed
- domains of original function is deduced
- decision tree for original function is built
- note how junk is removed in step 3 (but undefined "..." as last case)

\subsection{Support for Higher-Order Functions}

TODO:
- currying with explicit application symbol --> function arguments are now
  just constants
- apply encoding on application symbol
- gotcha: one function can be used with several arities.

- optim: specialization (similar to monomorphization)

\section{Case Studies}
\label{sec:case-studies}

\subsection{Lazy Lists}

XXX

\subsection{Huffman Coding}

XXX

\section{Evaluation}
\label{sec:evaluation}

{\looseness=-1
In this section, we evaluate both the overall impact of the translation
introduced in Section~\ref{sec:encoding} and the performance of individual SMT
techniques.
%
We gathered 602 benchmarks from three sources, which we will refer to as
\isanun, \isa and \leon. 
The first source consists of 357 benchmarks coming from\ajr{where Isa-Nun benchmarks come from, what they represent, are they expected to be unsat/sat}.
The second source consists of the 79
benchmarks from the IsaPlanner
suite~\cite{DBLP:conf/itp/JohanssonDB10} that do not contain higher-order
functions. These benchmarks have been used recently as challenge problems for a
variety of inductive theorem provers. They heavily involve
recursive functions and are limited to a theory of algebraic datatypes
with a signature that contains uninterpreted function symbols over these datatypes. 
The third source consists of 166
benchmarks from the Leon repository,\footnote{%Available at
\url{https://github.com/epfl-lara/leon/}} which were constructed from
verification conditions about simple Scala programs. These benchmarks also
heavily involve recursively defined functions over algebraic datatypes, 
but cover a wide variety of additional theories, including bit vectors, arrays, and
both linear and nonlinear arithmetic. All benchmarks are in definitional form
with respect to a set of well-founded functions.
A majority of the benchmarks in the latter two sets are unsatisfiable.
}

For each of the 245 benchmarks in the \isa and \leon sets, 
we considered up to three randomly selected
mutated forms of its conjecture $\psi$. In particular, we considered unique
conjectures that are obtained as a result of swapping a subterm of $\psi$ at
one position with another of the same type at another position.
Note that benchmarks created in this way have a high likelihood of having
small, easy-to-find countermodels. In total, we considered 213 mutated forms of
conjectures from \isa and 427 mutated forms of conjectures from \leon. We will
call these sets \isam and \leonm, respectively.
Thus, our benchmark set consists of 1242 benchmarks (640 mutants plus the
original 602). 
Each of the 1242 benchmarks were generated in SMT lib version 2.5 format~\cite{smtlib25}. %ajr : although these don't require 2.5 features
For \isanun, \ajr{describe generation of smt2 benchmarks without encoding}
The others were translated to SMT lib version 2 by the Leon tool.

We additionally considered these 1242 benchmarks both before and after the translation~$\conv$.
For example, \isa contains 79 original benchmarks $\varphi$ and 79 translated
benchmarks $\conv(\varphi)$.
For the \isanun, \ajr{describe generation of smt2 benchmarks with encoding}.
All others were translated by \cvc 
by outputting the result of running the preprocessor from Section~\ref{ssec:cvc4} on each benchmark.

For solvers, we considered the SMT solver \ziii~\cite{de-moura-bjoerner-2008},
which runs heuristic methods for quantifier instantiation~\cite{MouraBjoerner07}
as well as methods for finding models for quantified formulas~\cite{GeDeM-CAV-09}.
We also considered five configurations of \cvc~\cite{barrett-et-al-2011},
which we will refer to as \cvcd, \cvcf, \cvcfe, and \cvcfm.
The configuration \cvcd runs heuristic 
and conflict-based techniques for quantifier instantiation~\cite{ReynoldsTinelliMoura14},
but does not include techniques for finding models.
The other configurations run the finite model
finding procedure due to Reynolds et al.\ \cite{ReyEtAl-1-RR-13,reynolds-et-al-2013}.
The configuration \cvcfe additionally incorporates heuristic quantifier instantiation as described in Section 2.3 of~\cite{reynolds-et-al-2013},
and \cvcfm incorporates the fairness scheme for monotonic sorts as described in Section~\ref{ssec:cvc4}.

\begin{figure}[t]
%ct I would not use the normal size in figure makes them look out of proportion
%\normalsize
%\small
\small
\centering
\begin{tabular}{l@{\kern1.5em}r@{\kern0.375em}r@{\kern1.5em}r@{\kern0.375em}r@{\kern1.5em}r@{\kern0.375em}r@{\kern1.5em}r@{\kern0.375em}r@{\kern1.5em}r@{\kern0.375em}r@{\,\,}}
  & \multicolumn{2}{c@{\kern1.5em}}{\phantom{0}\ziiib}     & \multicolumn{2}{c@{\kern1.5em}}{\phantom{0}\cvcd}     
  & \multicolumn{2}{c@{\kern1.5em}}{\phantom{0}\cvcf}      & \multicolumn{2}{c@{\kern1.5em}}{\cvcfe} 
  & \multicolumn{2}{c@{\,\,}}{\cvcfm}
\\%[-1pt]
  & \hfill $\varphi$ \hfill & $\conv(\varphi)$\!\!
  & \hfill $\varphi$ \hfill & $\conv(\varphi)$\!\!
  & \hfill $\varphi$ \hfill & $\conv(\varphi)$\!\!
  & \hfill $\varphi$ \hfill & $\conv(\varphi)$\!\!
  & \hfill $\varphi$ \hfill & $\conv(\varphi)$\!\!
\\
\midrule

\isanun  & 0 & 14 & 0 & 0 & 0 & 155 & 0 & {\win 156} & 0 & 155
\\
\isa & 0 & 0 & 0 & 0 & 0 & 0 & 0 & 0 & 0 & 0
\\
\isam & 0 & 43 & 0 & 0 & 0 & {\win 161} & 0 & {\win 161} & 0 & {\win 161}
\\
\leon  & 8 & 30 & 6 & 6 & 6 & {\win 54} & 6 & 53 & 6 & {\win 54}
\\
\leonm & 6 & 61 & 0 & 0 & 2 & 180 & 2 & {\win 181} & 2 & {\win 181}
\\[\jot]
Total & 14 & 148 & 6 & 6 & 8 & 550 & 8 & {\win 551} & 8 & {\win 551}
\end{tabular}
\caption{\,Number of \emph{sat} responses on benchmarks without and with $\conv$ translation}
\label{fig:sat}
\end{figure}

\begin{figure}[t]
\small
%\normalsize
%\small
\centering
\begin{tabular}{l@{\kern1.5em}r@{\kern0.375em}r@{\kern1.5em}r@{\kern0.375em}r@{\kern1.5em}r@{\kern0.375em}r@{\kern1.5em}r@{\kern0.375em}r@{\kern1.5em}r@{\kern0.375em}r@{\,\,}}
  & \multicolumn{2}{c@{\kern1.5em}}{\phantom{0}\ziiib}     & \multicolumn{2}{c@{\kern1.5em}}{\phantom{0}\cvcd}     
  & \multicolumn{2}{c@{\kern1.5em}}{\phantom{0}\cvcf}      & \multicolumn{2}{c@{\kern1.5em}}{\cvcfe} 
  & \multicolumn{2}{c@{\,\,}}{\cvcfm}
\\%[-1pt]
  & \hfill $\varphi$ \hfill & $\conv(\varphi)$\!\!
  & \hfill $\varphi$ \hfill & $\conv(\varphi)$\!\!
  & \hfill $\varphi$ \hfill & $\conv(\varphi)$\!\!
  & \hfill $\varphi$ \hfill & $\conv(\varphi)$\!\!
  & \hfill $\varphi$ \hfill & $\conv(\varphi)$\!\!
\\
\midrule
\isanun & {\win 41} & 40 & 28 & 34 & 26 & 39 & 26 & 39 & 26 & 39
\\
\isa & 14 & {\win 15} & {\win 15} & {\win 15} & 1 & {\win 15} & {\win 15} & {\win 15} & 1 & {\win 15}
\\
\leon & 54 & 55 & {\win 56} & {\win 56} & 13 & 55 & {\win 56} & 55 & 13 & 55
\\
\isam & 48 & 52 & {\win 58} & 52 & 12 & 53 & {\win 58} & 50 & 12 & 53
\\
\leonm & 94 & 107 & 106 & 107 & 39 & {\win 108} & 106 & {\win 108} & 39 & {\win 108}
\\[\jot]
Total & 251 & 269 & 263 & 264 & 91 & {\win 270} & 261 & 267 & 91 & {\win 270}
\end{tabular}
\caption{\,Number of \emph{unsat} responses on benchmarks without and with $\conv$ translation}
\label{fig:unsat}
\end{figure}

The results are summarized in Figures \ref{fig:sat} and \ref{fig:unsat}.
The benchmarks and more detailed results are available online.%
\footnote{\url{http://lara.epfl.ch/~reynolds/IJCAR2016-recfun/}}
The figures are divided into benchmarks triggering \emph{unsat} and \emph{sat}
responses and further into benchmarks before and after the translation $\conv$.
The raw evaluation data reveals no cases in which a solver answered
\emph{unsat} on a benchmark $\varphi$ and \emph{sat} on its
corresponding benchmark $\conv( \varphi )$, or vice versa.
This is consistent with our expectations and Theorem~\ref{thm:equi}, 
since these benchmarks contain only well-founded function definitions.

Figure~\ref{fig:sat} shows that for untranslated benchmarks (the ``$\varphi$''
columns), the number of \emph{sat} responses is very low across all
configurations. This confirms the shortcomings of existing SMT techniques for
finding models for benchmarks containing recursively defined functions.
%
The translation $\conv$ (the ``$\conv(\varphi)$'' columns) has a major
impact. \cvcf finds 550 of the 1242 benchmarks to be satisfiable,
including 6~benchmarks in the nonmutated \leon benchmark set. 
The two optimizations for finite model finding in \cvc (configurations \cvcfe and \cvcfm) 
led to a net gain of 1 satisfiable benchmark each with respect to \cvcf.
The performance of \ziiib for countermodels also improved dramatically, as it
finds 134 more benchmarks to be satisfiable, including 5 that are not solved by \cvcf.
%
We conclude that the translation $\conv$ enables SMT
solvers to find countermodels for conjectures involving recursively defined functions
whose definitions are admissible.

Moreover,
the translation $\conv$ helps all configurations for \emph{unsat} responses as well.
\ziii solves a total of 269 with the translation, whereas it solves only 251 without it.
Surprisingly,
the configuration \cvcf, which is not tailored for handling unsatisfiable benchmarks,
solves 270 unsat benchmarks overall, which is more than both \cvcd and \ziii.
These results suggest that the translation do not degrade the performance 
of SMT solvers for unsatisfiable problems involving recursive functions, 
and instead often improve their performance.
\ajr{Importance of the translation helping unsat in practice for Isabelle/Nunchaku, if any}

\section{Related Work}

We described the most closely related work, by Ge and de Moura
\cite{GeDeM-CAV-09} and by Reynolds et al.\
\cite{ReyEtAl-1-RR-13,reynolds-et-al-2013}, in the text already.
The finite model finding support in the instantiation-based iProver
\cite{korovin-2013} is also close, given the similarities with SMT.

%Model finding has been studied outside the world of SMT.

Some finite model finders are based on a reduction to a decidable logic,
typically propositional logic. \begin{rep}They translate the input problem to the weaker
logic and pass it to a solver for that logic.\end{rep}
The translation is parameterized by upper or exact finite bounds on
the cardinalities of the atomic types. This procedure was pioneered by McCune
in the earlier versions of
\begin{conf}MACE\end{conf}\begin{rep}Mace (or MACE)\end{rep}
\cite{mccune-1994}.
Other conceptually similar finders are Paradox \cite{claessen-sorensson-2003}
and FM-Darwin \cite{baumgartner-et-al-2009} for first-order logic with
equality; the Alloy Analyzer and its back-end Kodkod \cite{torlak-jackson-2007}
for first-order relational logic; and Refute \cite{weber-2008} and Nitpick
\cite{blanchette-nipkow-2010} for higher-order logic.

An alternative is to perform
an exhaustive model search directly on the original problem. Given fixed
cardinalities, the search space is represented as multidimensional
tables. The procedure tries different values in the function and predicate
tables, checking each time if the problem is satisfied.
This approach was pioneered by FINDER
\cite{slaney-1994} and SEM \cite{zhang-zhang-1995}\begin{rep} and serves as
the basis of many more model finders, notably the Alloy Analyzer's precursor
\cite{jackson-1996} and the later versions of~Mace
\cite{mccune-prover9-mace4}\end{rep}.

\begin{rep}
Most of the above tools cannot cope with algebraic datatypes or other infinite
types. \end{rep}%
Kuncak and Jackson \cite{kuncak-jackson-2005} presented an idiom for
encoding datatypes and recursive functions in Alloy, by approximating datatypes
by finite subterm-closed substructures. The approach finds sound (fragments
of) models for formulas in the \relax{existential--bounded-universal} fragment%
\begin{rep} (i.e., formulas whose prenex normal forms contain no unbounded universal
quantifiers ranging over datatypes)\end{rep}. This idiom was further developed by Dunets
et al.\ \cite{dunets-et-al-2010}, who presented a translation scheme
for primitive recursion. Their definedness guards play a similar role to the
existential constraints generated by our translation $\conv$.
An approach related in scope to ours is given in~\cite{baumgartner2013},
which establishes the satisfiability of conjectures in the presence of admissible axioms over infinite domains
by proving their negation is entailed.

The higher-order model finder Nitpick \cite{blanchette-nipkow-2010}
for the Isabelle/HOL proof assistant
relies on another variant of Kuncak and Jackson's approach inside a
Kleene-style three-valued logic\begin{rep}, inspired by abstract interpretation\end{rep}.
\begin{rep}It was also the first tool of its kind to support corecursion and
coalgebraic datatypes \cite{blanchette-2013-relational}.\end{rep}
The three-valued logic approach extends each
approximated type with an unknown value, which is propagated by function
application. This scheme works reasonably well in Nitpick, because it builds
on a relational logic, but our initial experiments with \cvc suggest
that it is more efficient to avoid unknowns by adding existential
constraints.

The Leon system~\cite{blanc2013overview} implements a procedure that can
produce both proofs and counter\-examples for properties of terminating functions
written in a subset of Scala. Leon is based on an SMT solver. It avoids
quantifiers altogether by unfolding recursive definitions up to a certain
depth\begin{rep}, which is increased on a per-need basis\end{rep}.
Our translation~$\conv$ works in an analogous manner, 
where instead the SMT solver is invoked only once 
and quantifier instantiation is used in lieu of function unfolding.
\begin{rep}It would be
worth investigating how existing approaches for function
unfolding can inform approaches for dedicated quantifier instantiation
techniques for function definitions, and vice versa.\end{rep}

Model finding is concerned with satisfying arbitrary logical constraints. Some
tools are tailored for problems that correspond to total functional
programs. QuickCheck \cite{claessen-hughes-2000} for Haskell is an
early example, based on random testing. Bounded exhaustive testing\begin{rep}
\cite{runciman-et-al-2008}\end{rep} and narrowing\begin{rep}
\cite{lindblad-2008-testing}\end{rep} are other successful strategies. These
tools are often much faster than model finders, but they typically cannot cope
with underspecification and nonexecutable functions.

% others
%are Agsy for Agda \cite{xxx}, Quickcheck for Isabelle/HOL \cite{xxx}, and
% QuickChick for Coq \cite{xxx}.


%Unlike an SMT solver, Nitpick cannot rely on a built-in notion of
%a datatype; it axiomatizes finite subterm-closed substructures.

%\begin{verbatim}
%Nitpick
%  * nice: three-valued logic (e.g. local overflow)
%    * solving a different problem:
%      * no built-in notion, what's approximated is an idealized notion, no
%        injections $\concret$
%  * not so nice:
%    one abstract domain per type, not per function argument
%  * guards + one-domain per arg are a good combi, guards + one-single-domain is bad because of
%    odd schemes, give unsat example
%    [* show that it works also for example with weird recursion schemes]
%
%  * other features:
%    * inductive / coinductive predicates
%    * quotient types
%
%  * abstract interpretation
%
%Leon
%\end{verbatim}

\section{Conclusion}
\label{sec:conclusion}

We presented a translation scheme that extends the scope of finite model finding 
techniques in SMT, allowing one to use them to find models of quantified formulas 
over \relax{infinite} types, such as integers and algebraic datatypes.
%
In future work, it would be interesting to evaluate the approach against other
counterexample generators, notably Leon and Nitpick, and enrich the benchmark
suite with problems exercising \cvc's support for coalgebraic datatypes
\cite{reynolds-blanchette-2015-codata}. We also plan to integrate \cvc as a
counterexample generator in proof assistants.\ajr{Revise: this should say plans to develop Nunchaku.} Further work would also
include identifying additional sufficient conditions for admissibility, thereby
enlarging the applicability of the translation scheme presented here.

%\ct{?? Leon is a model finder? Alloy?}
%\jb{Alloy has no notion of datatypes, so it's pretty useless. Nitpick implements
%datatypes and a more FOL-like logic (actually, HOL) on top of Alloy's back-end,
%using the Kuncak \& Jackson idiom for Alloy, so that's as close as we can get
%to evaluating Alloy itself. Leon is, in part, a counterexample generator.
%``Model finder'' is probably a stretch.}

{%\footnotesize
\def\ackname{Acknowledgment}
\paragraph{%\footnotesize
\ackname.}
Viktor Kuncak and Stephan Merz have made this work possible. We would also like
to thank Damien Busato-Gaston and Emmanouil Koukoutos for providing the
set of Leon benchmarks used in the evaluation, and the anonymous reviewers for their
suggestions and comments.
}
%

{
\bibliographystyle{abbrv}
\bibliography{bib}
}

\end{document}
