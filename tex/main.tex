\documentclass[runningheads,a4paper]{llncs}
\usepackage[T1]{fontenc}
\usepackage[scaled=.82]{beramono}
\usepackage[scaled=.86]{helvet}
\usepackage{mathptmx}
\usepackage{listings}
\usepackage{colonequals}
%\usepackage{mathpartir}
\usepackage{xspace}
\usepackage{amssymb}
\setcounter{tocdepth}{3}
\usepackage{graphicx}
\usepackage{stmaryrd}
\usepackage{cite}
\usepackage{hyperref}
\usepackage{framed}

% http://mirror.switch.ch/ftp/mirror/tex/macros/latex/contrib/xcolor/xcolor.pdf
\usepackage{xcolor}

%\usepackage{amsthm}
\usepackage{amsmath}    % need for subequations
\usepackage{booktabs}
%\usepackage{float}
%\usepackage{fullpage}
%\floatstyle{ruled}
%\newfloat{alg}{ht}{loa}
%\floatname{alg}{Algorithm}

\usepackage{url}
\urldef{\mailsa}\path|firstname.lastname@epfl.ch|
\newcommand{\keywords}[1]{\par\addvspace\baselineskip
\noindent\keywordname\enspace\ignorespaces#1}

\DeclareFontFamily{OT1}{pzc}{}
\DeclareFontShape{OT1}{pzc}{m}{it}{<-> s * [1.10] pzcmi7t}{}
\DeclareMathAlphabet{\mathcalx}{OT1}{pzc}{m}{it}

\newcommand{\con}[1]{\mathsf{#1}}

\renewcommand\vec[1]{\overline{#1}}

\let\oldSigma=\Sigma
\def\Sigma{\mathrm{\oldSigma}}

\let\oldneg=\neg
\def\neg{\oldneg\:}

\newcommand{\cvc}{\textsc{cvc}{\small 4}\xspace}
\newcommand{\cvciii}{\textsc{cvc}{\small 3}\xspace}
\newcommand{\ziii}{\textsc{z}{\small 3}\xspace}
\newcommand{\teq}{\approx}
\newcommand{\cc}[1]{#1^*}

%%% I'm trying to avoid bold whenever it can be avoided (except for headings
%%% etc.) --JB
%\newcommand{\terms}{\mathbf{T}}
\newcommand{\terms}{\mathcalx{T}}

%\newcommand{\functerms}{\mathbf{DFT}}
\newcommand{\vars}{\mathbf{V}}
\newcommand{\M}{\mathcalx{M}}
\newcommand{\I}{\mathcalx{J}} %%% there's no difference, right? --JB
%% (I find \mathcalx{I} ugly, with its two horizontal bars, hence the J)
\newcommand{\J}{\mathcalx{J}}
\def\AIF{\qtab\keyword{if}\ }
\def\THEN{\ \keyword{then}\ }
\def\AELSE{\untab\qtab\keyword{else}\ }
\def\FI{\untab}
\def\RETURN{\keyword{return}\ }
\def\ENDPROC{\untab}
\newcommand{\conv}{\mathcalx{W}}

%\newcommand{\ssorts}[1]{#1^\mathrm{s}}
\newcommand{\ssorts}[1]{#1^\mathrm{y}}

\newcommand{\sfuns}[1]{#1^\mathrm{f}}
\newcommand{\sfundefs}[1]{#1^\mathrm{df}}
\newcommand{\sfunndefs}[1]{#1^\mathrm{nf}}
\newcommand{\sortint}{\mathrm{Int}}

\newcommand{\pnone}{\con{none}}
\newcommand{\ptrue}{\con{true}}
\newcommand{\pfalse}{\con{false}}
\newcommand{\pol}{\con{pol}}

%\newtheorem{remark}{Remark}

\newcommand{\Bool}{\con{Bool}}
\newcommand{\ltrue}{\top}
\newcommand{\lfalse}{\bot}
\newcommand{\lite}{\con{ite}}

\newcommand{\boolop}{\oplus}
\newcommand{\forallf}[1]{\forall_{\!#1\:}}
\newcommand{\fnull}{\emptyset}
\newcommand{\farg}[1]{a_{#1}}
\newcommand{\vecfarg}[1]{\vec{a}_{#1}}
\newcommand{\fargsort}[1]{\upsilon_{#1}}

\newcommand{\Sigmalia}{\Sigma_{LIA}}
\newcommand{\extendsig}[1]{\mathcalx{E}( #1 )}

\newcommand{\rem}[1]{\textcolor{red}{[#1]}}
\newcommand{\ajr}[1]{\rem{#1 --ajr}}
\newcommand{\jb}[1]{\rem{#1 --jb}}
\newcommand{\ct}[1]{\rem{#1 --ct}}

\newcommand{\vthinspace}{\kern+0.083333em}
\newcommand{\typ}[1]{^{\vthinspace #1}}

\newcommand{\definefunreccmd}{\con{define}\text{-}\con{fun}\text{-}\con{rec}}
\newcommand{\definefunsreccmd}{\con{define}\text{-}\con{funs}\text{-}\con{rec}}

\newcommand{\Mo}{{\mathcal{\!J\!}}}

%\newcommand{\euf}{\ensuremath{\mathrm{E}}\xspace}
\newcommand{\euf}{\ensuremath{\mathcalx{UF}}\xspace}
%\newcommand{\ari}{\ensuremath{\mathrm{A}}\xspace}
\newcommand{\ari}{\ensuremath{\mathcalx{A}}\xspace}


%\input{scalalistings}
%\input{smtlib2listings}
\usepackage{program}

\begin{document}

%%% I added (Co) -- hope you like it --JB
\title{Model Finding for (Co)recursive Functions in SMT\thanks{%
This research is partially supported by the Inria technological development
action ``Contre-exemples Utilisables par Isabelle et Coq'' (CUIC).
}
}

\author {Andrew Reynolds\inst{1} \and Jasmin Christian Blanchette\inst{2,3} \and Cesare Tinelli \inst{4}}
\authorrunning {A. Reynolds \and J. C. Blanchette \and C. Tinelli}
\institute{
\'Ecole Polytechnique F\'ed\'erale de Lausanne (EPFL), Switzerland
\and
Inria Nancy \& LORIA, Villers-l\`es-Nancy, France
\and
Max-Planck-Institut f\"ur Informatik, Saarbr\"ucken, Germany
\and
Department of Computer Science, University of Iowa, USA
}

\maketitle

\begin{abstract}
To do.
\end{abstract}

\section{Introduction}
\label{sec:introduction}

  * why model finding?
  * but SMT solvers better at proving
    * a single universal quantifier in an axiom is enough to put it off track
      (``maybe unsat'')

  * quantifiers in recursive fun defs

  * intuitively, finite quantification suffices:
    * if we make sure all needed values are there
    * if well-founded (assumed!)

  * nasty example (not well-founded)
    * f x = f x + 1

  * how we do it

  * show that it works also for example with weird recursion schemes

  * corecursive functions???
    * productive


\ajr{Demand for recursive functions in SMT (new draft includes define-funs-rec).  Usually not interested in showing logical consistency, there exist other tools that check e.g. termination. 
Thus, want SMT solver to check satisfiability the assumption that recursive functions are consistent (where this needs to be defined precisely).
}

Recent techniques~\cite{GeDeM-CAV-09, ReyEtAl-1-RR-13} 
have focused on finding models for quantified formulas in SMT.
Ge and de Moura \cite{GeDeM-CAV-09} presented a complete instantiation-based
procedure for formulas in the \emph{essentially uninterpreted} fragment.
This fragment is limited to universally quantified formulas where all
occurrences of variables occur as direct subterms of uninterpreted
functions. For example, the formula $\forall x%\typ{\sortint} --- the sort is not important, right?
.\;\, \con{f}( x )
\teq \con{g}( x ) + 5$ is in the essentially uninterpreted fragment.
% since all
% occurrences of $x$ occur beneath uninterpreted functions $\con{f}$ and
% $\con{g}$.
Other syntactic criteria are identified in this paper that extend
this fragment slightly, including cases when variables occur as arguments of
arithmetic predicates. The techniques in~Reynolds et al.\
\cite{ReyEtAl-1-RR-13} can be used for finding finite models for quantified
formulas over uninterpreted types as well as types having a fixed finite
interpretation, such as fixed-width bitvectors and enumerated datatypes.
Finite model finding techniques can
find a model for the formula $\forall x\typ{\tau} y\typ{\tau}.\;\, x \teq
y \vee \neg \con{f}( x ) \teq \con{f}( y )$ where the uninterpreted type $\tau$ is
interpreted as a finite set and $\con{f}$ as injective finite map.

Unfortunately, these fragments cannot accommodate the vast majority of
quantified formulas that correspond to recursive function definitions.
In particular, the fragments in~\cite{GeDeM-CAV-09} are not applicable when the argument of a recursive function is used in various ways (e.g. within an arithmetic expression), 
while the fragment in~\cite{ReyEtAl-1-RR-13} is not applicable for functions over infinite domains such as the integers.
So, for instance, the quantified formula $\forall x\typ{\sortint}.\;\, \lite( p \leq 0, p( x ) \teq 0, p( x ) \teq 2 * p( x - 1 ) )$ 
is not in the essentially uninterpreted fragment (since $x$ occurs as a child of $-$),
nor does finite model finding apply (since $x$ is of type $\sortint$).

Here, we advocate an approach that translates formulas involving function
definitions (such as the one mentioned in the previous paragraph)
into equisatisfiable \ct{to be defined} formulas where the
techniques~\cite{GeDeM-CAV-09, ReyEtAl-1-RR-13} are applicable.

\paragraph{Contributions}

In this paper, we make the following contributions:
\begin{itemize}
\item[-] We define a satisfiability-preserving translation method for a class of formulas containing recursive function definitions
that produces formulas for which existing model finding techniques~\cite{GeDeM-CAV-09, ReyEtAl-1-RR-13} are applicable.
\item[-] We provide experimental evidence that this translation improves the effectiveness of SMT solvers \cvc and \ziii for finding counterexamples to verification conditions.
\item[-] We detail front-end support for recursive function definitions and this translation in the SMT solver \cvc.
\end{itemize}

\section{Preliminaries}
\label{sec:prelim}

A \emph{signature} $\Sigma$ consists of 
a set $\ssorts{\Sigma}$ of type symbols (or sorts) and
a set $\sfuns{\Sigma}$ of function symbols.
We assume that signatures always include a Boolean type $\Bool$ and constants 
$\ltrue$ (truth) and $\lfalse$ (falsity) of that type,
an infix equality predicate ${\teq} : \tau \times \tau \to \Bool$
for each $\tau \in \ssorts{\Sigma}$,
standard boolean connectives ($\neg$, $\wedge$, $\vee$, etc.),
and a function $\lite : \Bool \times \tau \times \tau \rightarrow \tau$ for each $\tau \in \ssorts{\Sigma}$.
We write $t\typ{\tau}$ to denote a term $t$ of type~$\tau$ and
$\terms( t )$ to denote the set of subterms in $t$.
%We write $\functerms^\Sigma( t )$ to denote the set of $f$-applications in $t$ such that $f \in \sfundefs{\Sigma}$.

A \emph{$\Sigma$-interpretation $\I$} %is a mathematical structure that
maps each $\tau \in \ssorts{\Sigma}$ to a nonempty set $\tau^\I$,
the \emph{domain} of~$\tau$ in $\I$,
and each $\con{f} : \tau_1 \times \cdots \times \tau_n \rightarrow \tau$ in
$\sfuns{\Sigma}$ 
to a total function $\con{f}^\I : \tau_1^\I \times \cdots \times \tau_n^\I \rightarrow \tau^\I$.
A \emph{theory} is a pair $T = (\Sigma, \Mo)$ where 
$\Sigma$ is a signature and  $\Mo$ is a class of $\Sigma$-interpretations,
the \emph{models} of $T$.
A $\Sigma$-formula $\varphi$ is 
\emph{$T$-satisfiable}
if it is satisfied by some interpretation in $\Mo$;
otherwise, it is \emph{$T$-unsatisfiable}.
Formulas $\varphi$ and $\psi$ are \emph{equisatisfiable} if
$\varphi$ is $T$-satisfiable if and only if $\psi$ is $T$-satisfiable.
Given a signature $\Sigma$,
the \emph{theory of equality with uninterpreted functions} \euf
consists of the set of all $\Sigma$-interpretations.
We refer to the type and function symbols in this theory as \emph{uninterpreted}.

In this paper, we consider \emph{annotated quantifed formulas} of the form
$\forallf{a} \vec x.\; \varphi$, where the annotation~$a$ is either
an uninterpreted function $\con{f} \in \sfuns{\Sigma}$ or the special symbol
$\fnull$.
Given $\con{f} : \tau_1 \times \ldots \times \tau_n \rightarrow \tau$,
an annotated
quantified formula $\forallf{\con{f}} \vec x.\; \varphi$ is a \emph{function definition}
(\,\emph{for $\con{f}$}\vthinspace) if $\vec x$ is a tuple of variables of type
$\tau_1, \ldots, \tau_n$ and $\varphi$ is a (quantifier-free) formula equivalent to
$\con{f}( \vec x ) \teq t$ for some $t$. We write $\forall \vec
x.\; \varphi$ as shorthand for $\forallf{\fnull} \vec x.\; \varphi$, and $\exists
\vec x. \varphi$ as shorthand for $\neg \forall \vec x.\; \neg \varphi$.
For a formula $\varphi$, 
we write $\varphi[\vec t/\vec x]$ to denote the result of replacing all occurrences of $\vec x$ with $\vec t$.

\begin{definition}%
\rm
A formula $\varphi$ is in \emph{definitional form with respect to} functions $\{
\con{f}_1, \ldots, \con{f}_n \}$ if it is of the form
%
%\kern-\abovedisplayskip
%\kern+\abovedisplayshortskip
%
\begin{equation} \label{eq:wdf}
(\forallf{\con{f}_1} \vec x_1.\; \varphi_1) \wedge \ldots \wedge (\forallf{\con{f}_n} \vec x_n.\; \varphi_n) \wedge \psi
\end{equation}
%
where $\con{f}_1, \ldots, \con{f}_n$ are not $\fnull$ and distinct, 
and $\psi$ contains no function definitions.
We call $\psi$ the \emph{conjecture} of $\varphi$.
%\ct{Don't we need additional restrictions on the $\varphi_i$'s here?
%(E.g., they are quantifier-free, or perhaps existential or containing only let binders.)}
% Now fixed above -ajr
\end{definition}

Given a signature $\Sigma$,
we distinguish the subset of \emph{defined} function symbols $\sfundefs{\Sigma}
\subseteq \sfuns{\Sigma}$.
We consider $\Sigma$-formulas that are in definitional form with respect to
functions $\sfundefs{\Sigma}$.

\begin{definition}
Given a set of function definitions $\mathcal{D} = \{ \forallf{f_1} \vec x. \varphi_1, \ldots, \forallf{f_n} \vec x. \varphi_n \}$,
a formula $\varphi$ is \emph{closed under function expansion with respect to $\mathcal{D}$} if
\begin{eqnarray}
\varphi \text{ entails } \displaystyle\bigwedge\limits_{i=1}^n \{ \varphi_i[ \vec t / \vec x ] \mid f_i( \vec t ) \in \terms( \varphi ) \}
\end{eqnarray}
The set $\mathcal{D}$ is \emph{admissible} if whenever there exists a model for $\varphi$ that is closed under function expansion,
there exists a model for $\varphi \wedge \forallf{f_1} \vec x. \varphi_1 \wedge \ldots \wedge \forallf{f_n} \vec x. \varphi_n$.
\end{definition}

For instance, the set $\{ \forallf{f} \vec x. f( x ) \teq 0 \}$ is admissible.
A set of admissible function definitions can include definitions of non-terminating functions, e.g.
$\{ \forallf{f} x^\sortint. f( x ) \teq f( x - 1 ) \}$ or even $\{ \forallf{f} x. f( x ) \teq f( x ) \}$ are admissible.
On the other hand, the set $\{ \forallf{f} x. f( x ) \teq f( x ) + 1 \}$ is not admissible, 
since the empty set of formulas is closed under function expansion with respect to this set,
and there is no model satisfying $\forallf{f} x. f( x ) \teq f( x ) + 1$.
Consider the set $\{ \forallf{f} x. f( x ) \teq f( x ), \forallf{g}. \lite( \neg f( x ) \teq 5, g( x ) \teq g( x ) + 1, g( x ) \teq f( x ) ) \}$.
While this set has a model where $f$ and $g$ are interpreted as the constant $5$, it is not admissible
since the formula $f( 0 ) \teq 4$ is closed under function expansion with respect to this set,
but there exists no interpretation satisfying both $f( 0 ) \teq 4$ and $\forallf{g}. \lite( \neg f( x ) \teq 5, g( x ) \teq g( x ) + 1, g( x ) \teq f( x ) )$.


\jb{TODO: Jasmin: Identify fragments: terminating recursive functions, etc.}

%Terminating function specifications are always admissible, but also some
%nonterminating
%An admissible function definition need not specify a terminating function, e.g.
%$\{ \forallf{\con{f}} x^\sortint.\;\, \con{f}( x ) \teq \con{f}( x - 1 ) \}$ or even $\{ \forallf{\con{f}} x.\;\, \con{f}( x ) \teq \con{f}( x ) \}$ are admissible.

%\section{A Translation for Admissible Function Definitions}
%\section{A Translation for Function Definitions}
\section{The Translation}
\label{sec:encoding}

Let us fix a signature $\Sigma$, a $\Sigma$-formula
$\varphi$ in definitional form with respect to functions $\sfundefs{\Sigma}$,
whose definitions are admissible.
%We may translate $\varphi$ into a equisatisfiable formula $\psi$ for which known model-finding procedures~\cite{GeDeM-CAV-09, ReyEtAl-1-RR-13} are applicable.
This section presents a method for constructing an extended signature
$\extendsig{ \Sigma }$ and a $\extendsig{ \Sigma }$-formula $\psi$ such that
$\varphi$ and $\psi$ are equisatisfiable.
\jb{Equisatisfiability as defined in Sect.\ 2 requires the same $\Sigma$ and
the same $T$ for both formulas, which isn't the case here.}
The intuition behind this translation
is to use an uninterpreted type \ct{to be defined} to abstract the set of
\emph{relevant} tuples for each recursive function $\con{f}$ and restrict the
quantification of the function definition for $\con{f}$ to a variable of this
type. Informally, the relevant tuples $\vec t$ of a function $\con{f}$ are the
ones for which the interpretation of $\con{f}( \vec t )$ is relevant to the
satisfiability of $\varphi$.

\begin{figure}[t]
\begin{enumerate}
%\begin{framed}
\item[\ ] 
$\conv( \varphi )$ : 
 \begin{itemize}
   \item[] $\mathsf{let}$ $( \psi, D ) = \conv( \varphi, \ptrue )$;
   %\item[] $\mathsf{assert}( D = \emptyset )$;
   \item[] $\mathsf{return}$ $\psi$
  \end{itemize}
\end{enumerate}
\begin{enumerate}
%\begin{framed}
\item[\ ] 
$\conv_0( t\typ{\tau}, p )$ : 
 \begin{itemize}
   \item[] $\mathsf{if}$ $\tau \equiv \Bool$ and $t \equiv t_1 \boolop \ldots \boolop t_n$ :
    \begin{itemize}
      \item[] $\mathsf{let}$ $( r_i, D_i ) = \conv_0( t_i, \pol( \boolop, i, p ) )$ for $i = 1, \ldots, n$;
      \item[] $\mathsf{if}$ $p = \ptrue$
      \begin{itemize}
        \item[] $\mathsf{return}$ $( ( r_1 \boolop \ldots \boolop r_n ) \wedge D_1 \wedge \ldots \wedge D_n, \emptyset )$
      \end{itemize}
      \item[] $\mathsf{else}$ $\mathsf{if}$ $p = \pfalse$
      \begin{itemize}
        \item[] $\mathsf{return}$ $( ( r_1 \boolop \ldots \boolop r_n ) \vee \neg D_1 \vee \ldots \vee \neg D_n, \emptyset )$
      \end{itemize}
      \item[] $\mathsf{else}$
      \begin{itemize}
        \item[] $\mathsf{return}$ $( r_1 \boolop \ldots \boolop r_n, D_1 \cup \ldots \cup D_n )$
      \end{itemize}
    \end{itemize}
  \item[] $\mathsf{else}$ $\mathsf{if}$ $t \equiv \forallf{\con{f}} \vec x.\;\, t_1$
    \begin{itemize}
      \item[] $\mathsf{let} ( r_1, D_1 ) = \conv_0( t_1, p )$;
      \item[] $\mathsf{if}$ $\con{f} \in \sfundefs{\Sigma}$
      \begin{itemize}
        %\item[] $\mathsf{assert}( D_1 = \emptyset )$;
        \item[] $\mathsf{return}$ $( \forall y\typ{\fargsort{\con{f}}}.\;\, ( r_1 [ \vecfarg{\con{f}}( y ) / \vec x ] ), \emptyset )$
      \end{itemize}
      \item[] else
      \begin{itemize}
        \item[] $\mathsf{return}$ $( \forall \vec x.\; r_1, \forall \vec x.\; D_1 )$
      \end{itemize}
    \end{itemize}
   \item[] $\mathsf{else}$
   \begin{itemize}
     \item[] $\mathsf{return}$ $( t, \{ \exists z\typ{\fargsort{\con{f}}}. ( \vecfarg{f}( z ) \teq \vec t ) \mid f( \vec t ) \in \terms( t ), f \in \sfundefs{\Sigma} \} )$
   \end{itemize}
 \end{itemize}
\end{enumerate}
\vspace{-2ex}
\caption{A translation procedure $\conv$ for a $\Sigma$-formula that is in definitional form with respect to a set of admissible function definitions $\sfundefs{\Sigma}$.
In the subprocedure $\conv_0$, the procedure $\pol( \boolop, i, p )$ returns the polarity of the $i^{th}$ child of a $\boolop$-application having polarity $p$,
where $\boolop$ is an interpreted predicate.
}
\label{fig:encoding}
\end{figure}

In more detail, 
say our signature $\Sigma$ contains a set of defined function symbols $\sfundefs{\Sigma} \subseteq \sfuns{\Sigma}$.
We extend the signature of $\Sigma$ to a signature $\extendsig{\Sigma}$ containing the following additional definitions.
For each $\con{f} \in \sfundefs{\Sigma}$ of type $\tau_1 \times \ldots \times \tau_n \rightarrow \tau$, 
the signature $\extendsig{\Sigma}$ includes:
\begin{enumerate}
\item[-] an uninterpreted type $\fargsort{\con{f}}$, and
\item[-] a vector of $n$ uninterpreted functions $\farg{\con{f}}^1$, $\ldots$, $\farg{\con{f}}^n$ of type $(\fargsort{\con{f}} \rightarrow \tau_1)$, $\ldots$, $(\fargsort{\con{f}} \rightarrow \tau_n)$.
\end{enumerate}
%The interpretation of uninterpreted sort $\fargsort{\con{f}}$ will denote the elements (tuples) on which the function $\con{f}$ is applied.
%The role of the uninterpreted functions $\farg{\con{f}}^1$, $\ldots$, $\farg{\con{f}}^n$ will be discussed more in the following.

Given the extended signature, we run the procedure $\conv$ in Figure~\ref{fig:encoding} on $\varphi$.
This calls the subprocedure $\conv_0$, which takes two arguments: the term $t$ to translate, and a \emph{polarity} $p$ (either $\ptrue$, $\pfalse$, or $\pnone$).
It returns a pair of the form $( r, D )$, where $r$ is a term of type $\tau$, and $D$ is a set of formulas.
%The role of $D$ is to ensure that the (restricted) function definition for $\con{f}$ includes certain tuples in its domain, which we explain more in the following.
At a high level, the translation modifies $\varphi$ in two ways.
First, it restricts the quantification on function definitions for $\con{f}$ to the corresponding uninterpreted type $\fargsort{\con{f}}$.
Second, it augments $\varphi$ with additional constraints of the form $\exists z\typ{\fargsort{\con{f}}}. ( \vecfarg{\con{f}}( z ) \teq \vec t )$
which ensure that this restriction for the defintion $\con{f}$ covers all relevant tuples of terms,
namely the ones for which an $\con{f}$-application exists and is relevant to the satisfiability of the current branch of $\varphi$.

In the case that $t$ is an application of (interpreted) predicate $\boolop$ (e.g., a boolean connective, or equality),
we recursively call $\conv$ on its subchildren $t_i$ and polarity $\pol( \boolop, i, p )$, where $\pol$ is defined as\footnote{The negation $\neg p$ of polarity $p$ is 
$\pfalse$ if $p$ is $\ptrue$,
$\ptrue$ if $p$ is $\pfalse$, and
$\pnone$ if $p$ is $\pnone$.}:
\begin{equation*}
\pol( \boolop, i, p ) = \begin{cases}
                         \pnone & \boolop \equiv \teq \text{ or } ( \boolop \equiv \lite \text{ and } i=0 ) \\
                         \neg p & \boolop \equiv \neg \\
                         p & \text{otherwise}
                         \end{cases}
\end{equation*}
If $t$ has positive polarity,
we construct a conjunction of the result of the recursive calls with formulas $D_1, \ldots, D_n$.
Similarly, if $t$ has negative polarity,
we construct a disjunction of the result of the recursive calls with the negation of formulas $D_1, \ldots, D_n$.
It $t$ has no polarity,
then we return the union of $D_1 \cup \ldots \cup D_n$.

In the case that $t$ is a quantified formula $\forallf{\con{f}} \vec x.\; t_1$, it recursively calls $\conv$ on its body with the same polarity.
In the case that $t$ is a function definition of a function in $\sfundefs{\Sigma}$,
we instead construct a quantified formula over a single variable $y$ of type $\fargsort{\con{f}}$,
and replace all occurrences of $\vec x$ in $r_1$ with $\vecfarg{\con{f}}( y )$.
In this case, since function definitions are a top-level conjunct, by case analysis on the return values of $\conv$, we know that $D_1$ is empty.
In the case that $t$ is not a function definition, 
the resulting quantified formula is reconstructed with the term $r_1$ returned by the recursive call,
and a quantified prefix is appended to the formula $D_1$ returned by the recursive call.

Otherwise, $t$ is either an application of an uninterpreted predicate, or a term of a type other than $\Bool$.
In this case, the second component of our return value contains a constraint of the form $\exists z\typ{\fargsort{\con{f}}}. ( \vecfarg{\con{f}}( z ) \teq \vec t )$
for each subterm $\con{f}( \vec t )$ of $t$ such that $\con{f}$ is a function in $\sfundefs{\Sigma}$.
When such constraints are asserted positively, this indicates that the tuple of uninterpreted functions $\vecfarg{\con{f}}$ must include $\vec t$ in its range.

We demonstrate this translation with an example.

\begin{example}
\label{ex:translation}
Consider the (combined) signature of linear arithmetic and uninterpreted functions $\Sigma = \Sigmalia \cup \Sigma_{\euf}$,
where $\ssorts{\Sigma_{\euf}} = \{ \sortint \}$ and 
$\sfuns{\Sigma_{\euf}} = \{ s\typ{ \sortint \rightarrow \sortint }, c\typ{ \sortint } \}$.
Consider the $\Sigma$-formula $\varphi$:
\begin{equation} \label{eq:ex-before}
\forall_{s} x\typ{\sortint}.\; \lite( x \leq 0, s(x) \teq 0, s( x ) \teq x + s( x-1 ) ) \wedge s( c ) > 100
\end{equation} 
Here, $s$ is a function that returns the sum of positive numbers between $0$ and its argument $x$.
The formula $\varphi$ is in definitional form with respect to functions $\sfundefs{\Sigma_u}$,
and states that the sum of all positive numbers between $0$ and $c$ is greater than $100$.
This formula is satisfiable with a model that interprets $c$ as an integer greater than or equal to $14$.
Due to the universally quantified formula in the left disjunct,
current SMT techniques~\cite{GeDeM-CAV-09, ReyEtAl-1-RR-13} are unable to find a model for this formula.
The signature $\extendsig{\Sigma}$ includes the type $\fargsort{s}$,
and the uninterpreted function $\farg{s}$ of type $\fargsort{s} \rightarrow \sortint$.
The result of $\conv( \varphi, \ptrue )$ is the pair $( \psi, \emptyset )$, where after simplification, $\psi$ is the $\extendsig{\Sigma}$-formula:
\begin{eqnarray} \label{eq:ex-after}
\begin{split}
\forall y\typ{\fargsort{s}}.\;\, \lite( & \farg{s}( y ) \leq 0,  \\
 & s(\farg{s}( y )) \teq 0, \\
 & s(\farg{s}( y )) \teq \farg{s}( y ) + s( \farg{s}( y )-1 ) ) \\
 & \wedge \exists z\typ{\fargsort{s}}. \farg{s}( z ) \teq \farg{s}( y )-1 )
\end{split}  
\wedge ( s( c ) > 100 \wedge \exists z\typ{\fargsort{s}}. \farg{s}( z ) \teq c )
\end{eqnarray} 
The universal quantification in the resulting formula is over an uninterpreted type $\fargsort{s}$, 
thus enabling finite model finding techniques~\cite{ReyEtAl-1-RR-13} to search for a finite interpretation for $\fargsort{s}$.
Moreover, all occurrences of $y$ in this formula are beneath applications of the uninterpreted function $\farg{s}$,
implying that the formula is in the essentially uninterpreted fragment of~\cite{GeDeM-CAV-09}, for which there exists a complete instantiation procedure.
We observe that both \cvc and \ziii run indefinitely on formula~(\ref{eq:ex-before}),
and both produce a model for formula~(\ref{eq:ex-after}) in $<0.1$ second.
$\square$
\end{example}

It is important to note that the translation $\conv$ does not preserve the models of $\varphi$.
One model $\I$ for formula~(\ref{eq:ex-after}) in the previous example interprets 
$\fargsort{s}$ as a finite set of elements $\{ u_0, \ldots, u_{14} \}$,
$\farg{s}$ as a finite map mapping containing $u_i \mapsto i$ for $i = 0, \ldots, 14$,
$c$ as $14$, 
and $s$ as the (almost constant) function:
\begin{equation} \label{eq:approx-interp}
\lambda x\typ{Int}. \lite( x \teq 0, 0, \lite( x \teq 1, 1, \lite( x \teq 2, 3, \ldots, \lite( x \teq 13, 91, 105 ) \ldots )
\end{equation}
In other words, $s$ is a function mapping $x$ to the sum of all positive integers between $0$ and $x$ when $0 \leq x \leq 13$,
and $105$ otherwise.
Notice that $\I$ is \emph{not} a model for the original formula~(\ref{eq:ex-before}),
since $\I$ interprets e.g.\ $s( -1 )$ and $s( 15 )$ as $105$.
However, under the assumption that function definitions in $\sfundefs{\Sigma}$ are admissible, 
we claim that $\conv(\varphi)$ is equisatisfiable to $\varphi$ for any input $\varphi$.
The intuition is that the interpretation of a term such as $s( -1 )$ in the previous example is not relevant to the satisfiability of formula~(\ref{eq:ex-before}).
Thus, a satisfiable (resp. unsatisfiable) response from an SMT solver on input $\conv( \varphi )$ implies the existence (resp.\ nonexistence) of a model for $\varphi$.
Moreover, information in models for $\conv( \varphi )$ contains relevant information regarding the models of $\varphi$.
For example, the model $\I$ for formula~(\ref{eq:ex-after}) described above interprets $c$ as $14$, 
while there exists a model of formula~(\ref{eq:ex-before}) also interprets $c$ as $14$.
In fact, we claim that there exists a model of $\varphi$ that coincides with each model of $\conv( \varphi )$ on its interpretation of all functions in $\sfuns{\Sigma}$ 
other than those in $\sfundefs{\Sigma}$, i.e. $\{ s \}$ in the previous example.
From a practical point of view, this is not an issue, because
$s$ is the very function that was explicitly defined by the user,
and hence already has an intended interpretation.

\ajr{add invariants, proofs}

%\begin{lemma}
%Let $\M$ be a model satisfying $\conv( \varphi )$, and
%%and let $\conv( \varphi )^\ast$ be the quantifier-free formula obtained by 
%let $M$ be a set of literals such that $\M \models M$ and $M \models_p \conv( \varphi )$.
%For each assignment $\sigma$ over $\vars{ M }$,
%each $\con{f}( \vec t ) \in \terms( M \sigma )$ where $\con{f} \in \sfundefs{\Sigma}$, 
%there exists a formula of the form $\exists z\typ{\fargsort{\con{f}}}. ( \vecfarg{\con{f}}( z ) \teq \vec t )$ in $M \sigma$.
%\end{lemma}

\begin{theorem}
If $\varphi$ is a $\Sigma$-formula that is in definitional form with respect to functions $\sfundefs{\Sigma}$,
and the set of function definitions $\mathcal{D}$ corresponding to functions in $\sfundefs{\Sigma}$ is admissible,
then $\varphi$ and $\conv( \varphi )$ are equisatisfiable.
\end{theorem}
\begin{proof}
(Sketch)  First, we show that if $\varphi$ is satisfied by $\Sigma$-interpretation $\I$, then $\conv( \varphi )$ is satisfied by a $\extendsig{\Sigma}$-interpretation $\J$.
Let $\J$ be the $\extendsig{\Sigma}$-interpretation that interprets all types $\tau \in \ssorts{\Sigma}$ as $\tau^\I$,
all functions $\con{f} \in \sfuns{\Sigma}$ as $\con{f}^\I$,
and for each function $\con{f}$ of type $\tau_1 \times \ldots \tau_n \rightarrow \tau$ in $\sfundefs{\Sigma}$, 
interprets $\fargsort{\con{f}}$ as a tuple of values of type $\tau_1^\I \times \ldots \times \tau_n^\I$,
and each $\farg{\con{f}}^i$ as the $i^{th}$ projection on such tuples for each $i = 1, \ldots, n$.
Note that $\J$ satisfies \emph{every} constraint of the form $\exists z\typ{\fargsort{\con{f}}}. ( \vecfarg{\con{f}}( z ) \teq \vec t )$,
since by our construction of $\J$ there exists a $z$ such that $z^\J = \vec t^\J$.
By case analysis it follows that $\conv_0$ maintains the invariant that
if $\conv_0( t\typ{\Bool}, \ptrue )$ returns $( r\typ{\Bool}, D )$,
then $\I$ satisfies $t$ if and only if $\J$ satisfies $r$.
Thus, we conclude $\J$ is a model of $\conv( \varphi )$.

Second, we show that if $\conv( \varphi )$ is satisfied by a $\extendsig{\Sigma}$-interpretation $\I$, then $\varphi$ is satisfied by $\Sigma$-interpretation $\J$.
Since $\varphi$ is in definitional form with respect to functions defined by $\mathcal{D}$, we know $\varphi$ is of the form $\mathcal{D} \wedge \varphi_0$.
We first argue that $\I$ satisfies a $\Sigma$-formula $\psi$ that is closed under function expansion with respect to $\mathcal{D}$,
and where $\psi$ entails $\varphi_0$.

Since there exists a model for $\psi$ that is closed under function expansion with respect to $\mathcal{D}$, 
then since $\mathcal{D}$ is admissible,
by definition there exists a model for $\psi \wedge \mathcal{D}$, which entails $\mathcal{D} \wedge \varphi_0$, which is $\varphi$.
\end{proof}

Let us revisit the translated formula~(\ref{eq:ex-after}) from Example~\ref{ex:translation}.
This formula is unsatisfiable if and only if formula~(\ref{eq:ex-before}) is unsatisfiable since $\fargsort{f}$ can be interpreted as $\sortint$,
and $\farg{f}$ can be interpreted as the identity function.
Conversely, we claim formula~(\ref{eq:ex-after}) is satisfiable if and only if formula~(\ref{eq:ex-before}) is satisfiable,
noting that the set $\{ \forall_{s} x\typ{\sortint}. ite( x \leq 0, s(x) \teq 0, s( x ) \teq x + s( x-1 ) ) \}$ is admissble.
As stated in the proof of the previous theorem, 
the reason is that any model of formula~(\ref{eq:ex-after}) must satisfy a formula that is closed under function expansion,
hence implying that formula~(\ref{eq:ex-before}) has a model.
In more detail, let $\varphi_s$ be the formula $\forall_{s} x. ite( x \leq 0, s(x) \teq 0, s( x ) \teq x + s( x-1 ) )$.
The rightmost conjunct 
$\exists z\typ{\fargsort{s}}. \farg{s}( z ) \teq c$ ensures 
that any model of formula~(\ref{eq:ex-after}) satisfying $s( c ) > 100$ also satisfies $\varphi_s[c/x]$.
Similarly, the nested existential $\exists z. \farg{s}( z ) \teq \farg{s}( y )-1 )$
ensures that this constraint is enforced recursively.
%Given a model $\I$ of formula~(\ref{eq:ex-after}) that interprets $c$ as $14$, it is easy to see that $\I$ also satisfies the formula:
\begin{eqnarray}
\begin{split}
s( c ) > 100 \wedge & \displaystyle\bigwedge\limits_{i=0}^{13} ( \neg c-i \leq 0 \wedge s( c-i ) \teq i + s( c-i-1 ) ) \\
                    & \wedge c-14 \leq 0 \wedge s( c-14 ) \teq 0
\end{split}
\end{eqnarray}

\section{Evaluation}
\label{sec:evaluation}

In this section, we evaluate both 
the overall impact of the encoding introduced in the previous section, and
the performance of individual SMT techniques for benchmarks in the encoding.

For benchmarks, we gathered a set of 245 benchmarks from two sources, which we will refer to as {\bf isa} and {\bf leon}.
The first ({\bf isa}) consists of the 79 benchmarks from the Isaplanner benchmark suite~\cite{DBLP:conf/itp/JohanssonDB10} that do not contain higher-order functions.
The benchmarks have been used recently as challenge problems for a variety of inductive theorem provers.
These benchmarks heavily involve recursive functions, and are limited to the combined theory of uninterpreted functions and inductive datatypes.
The second ({\bf leon}) consists of 166 benchmarks from the Leon repository~\footnote{Available at \url{https://github.com/epfl-lara/leon}.},
which were constructed from verification conditions from simple Scala programs.
These benchmarks also heavily involve recursive functions over inductive datatypes, 
and also cover a wide variety of theories, including bitvectors, arrays, and linear and non-linear arithmetic.
Each of the 245 benchmarks are of the form described in formula~(\ref{eq:wdf}) from Section~\ref{sec:prelim},
in particular they each are a conjoined list of function definitions followed by a (negated) conjecture.
A majority of these benchmarks are unsatisfiable, although a handful of the benchmarks in the {\bf leon} set are satisfiable.

For each of these 245 benchmarks, we considered (up to) three randomly selected mutated forms of its conjecture $\psi$.
In particular, we considered unique conjectures that are obtained as a result of swapping a subterm of $\psi$ at one position
with another of the same type at another position.
Note that benchmarks created in this way have a high likelihood of having small and easy-to-find counterexamples.
In total, we considered 213 mutated forms of conjectures from {\bf isa}, and 427 mutated forms of conjectures from {\bf leon}.
We will call these sets {\bf isa+m} and {\bf leon+m} respectively.

In total, our benchmark set consisted of 885 benchmarks (640 mutants plus the original 245).
We considered these 885 benchmarks both before and after the translation $\conv$ as described in Section~\ref{sec:encoding}.
We will denote the latter encoding using the suffix {\bf w}.
For example, {\bf isa+w} contains 79 benchmarks that are the results of
applying $\conv$ to each of the 79 benchmarks in {\bf isa}.

For solvers, we considered the SMT solver \ziii~\cite{de-moura-bjoerner-2008}, 
which runs both heuristic methods for quantifier instantiation~\cite{DBLP:conf/cade/MouraB07},
as well as methods for finding models for quantified formulas~\cite{GeDeM-CAV-09}.
We also considered three configurations of \cvc~\cite{barrett-et-al-2011} which we will refer to as {\bf cvc4}, {\bf cvc4+f} and {\bf cvc4+i}.
The default configuration {\bf cvc4} runs heuristic quantifier instantiation, 
but does not include techniques for finding models.
The configuration {\bf cvc4+f} runs heuristic instantiation and the finite model finding procedure described in~\cite{ReyEtAl-1-RR-13, reynolds-et-al-2013}.
The last configuration {\bf cvc4+i} incorporates techniques for automating inductive reasoning in SMT~\cite{reynolds-kuncak-2015}.
%All configurations of \cvc incorporate techniques for conflict-based instantiation~\cite{} ..?

\begin{figure}[t]
\centering
{
\begin{tabular}{|l|cc|cc|cc|cc|l|cc|cc|cc|cc|}
\hline                                                                
  & \multicolumn{2}{c|}{{\bf z3}}     & \multicolumn{2}{c|}{{\bf cvc4}}     & \multicolumn{2}{c|}{{\bf cvc4+f}}     & \multicolumn{2}{c|}{{\bf cvc4+i}}     & & \multicolumn{2}{c|}{{\bf z3}}     & \multicolumn{2}{c|}{{\bf cvc4}}     & \multicolumn{2}{c|}{{\bf cvc4+f}}     & \multicolumn{2}{c|}{{\bf cvc4+i}}     
\\                                                                    
  & u & s & u & s & u & s & u & s & & u & s & u & s & u & s & u & s 
\\                                                                    
\hline                                                                    
{\bf isa} & 14  & 0 & 15  & 0 & 15  & 0 & {\bf 61}  & 0 & {\bf isa+w} & 15  & 0 & 15  & 0 & 15  & 0 & 14  & 0
\\                                                                  
{\bf leon}  & 73  & 0 & 80  & 0 & 80  & 0 & {\bf 96}  & 0 & {\bf leon+w}  & 78  & 2 & 80  & 0 & 76  & {\bf 9} & 78  & 0
\\                                                                   
{\bf isa+m} & 17  & 0 & 18  & 0 & 18  & 0 & {\bf 44}  & 0 & {\bf isa+mw}  & 18  & 35  & 18  & 0 & 18  & {\bf 153} & 17  & 0
\\                                                                  
{\bf leon+m}  & 83  & 11  & 103 & 6 & 104 & 6 & {\bf 117} & 6 & {\bf leon+mw} & 98  & 75  & 98  & 6 & 95  & {\bf 169} & 98  & 6
\\                                                                  
\hline                                                             
{\bf total} & 187 & 11  & 216 & 6 & 217 & 6 & {\bf 318} & 6 & {\bf total} & 209 & 112 & 211 & 6 & 204 & {\bf 331} & 207 & 6
\\                                                                                            
\hline                                                                                            
\end{tabular}
\\
}
\caption{Number of unsat and sat responses for of all configurations on benchmarks 
before and after the translation $\conv$.}
\label{fig:results}
\end{figure}

The results are shown in Figure~\ref{fig:results}.
The results are divided into benchmarks before and after the translation $\conv$.
First, we note that in no cases did a configuration answer differently across the two encodings.
In other words, there were no cases in which \ziii or \cvc answered e.g.\ ``unsatisfiable" on a benchmark $\varphi$
and ``satisfiable" on its corresponding benchmark $\conv( \varphi )$, or vice versa.
This provides confirming evidence that the translation $\conv$ is \emph{refutation-sound},
as we did not find a case where $\conv( \varphi )$ was unsatisfiable when $\varphi$ was satisfiable.
It also provides confirming evidence that the translation is \emph{model-sound} for this set of benchmarks,
as we did not find a case where $\conv( \varphi )$ was satisfiable when $\varphi$ was unsatisfiable.
This is to be expected, since the functions in our benchmark sets are expected to be admissible (and in fact, terminating).

Examining the first encoding in isolation (the first nine columns),
the number of satisfiable responses from all configurations is quite low.
In total, {\bf z3} manages to find only 11 benchmarks to be satisfiable,
whereas each configuration of \cvc only finds 6.
This confirms the shortcomings of existing SMT techniques for finding counterexamples for benchmarks containing recursive functions.
In terms of number of unsatisfiable responses,
the configuration {\bf cvc4+i} is the clear winner, finding 318 total benchmarks to be unsatisfiable.

Now, examining the second encoding (the latter nine columns), 
we see that using the translation $\conv$ drastically improves the ability of configurations for answering satisfiable.
In total, {\bf cvc4+f} finds 331 of the 885 benchmarks to be satisfiable, including 9 benchmarks in the original {\bf leon} benchmark set.
The performance of {\bf z3} for counterexamples also improves drastically, as it solves 112 satisfiable,
including 10 that were unsolved by {\bf cvc4+f}.
Moreover, the translation $\conv$ helps \ziii for unsatisfiable responses as well,
as we see \ziii solves a total of 209 whereas it solves only 187 before the translation.
This number is 2 fewer than {\bf cvc4}, which solves 211 unsatisfiable.
Thus, in most cases the translation $\conv$ does not significantly degrade performance for unsatisfiable benchmarks,
and in some cases may actually aid the solver for determining unsatisfiability.%
\footnote{The exception here is {\bf cvc4+i}, which makes use of type
information associated with quantified formulas for inductive arguments.}

We conclude from this evaluation that the translation $\conv$ enables SMT solvers to find counterexamples 
for conjectures involving recursive functions whose definitions are admissible.
Both {\bf z3} and {\bf cvc4+f} were able to find a large number of counterexamples after the translation,
with the latter configuration finding significantly more instances (331 to 112).

\section{Front-End Support for Recursive Function Definitions in CVC4}
\label{sec:front-end}

We have seen a satisfiability-preserving translation $\conv$ that significantly improves the effectiveness of known SMT techniques
for finding counterexamples for formulas involving recursive function definitions.
For convienience, the translation $\conv$ has been implemented in the latest development version of \cvc~\cite{ReyEtAl-1-RR-13} (version 1.5 pre-release).
Function definitions $\forallf{\con{f}} \vec x.\; \varphi$ can be written in using the $\definefunreccmd$ command, 
from Version 2.5~\cite{} of the SMT-LIB standard.
For example, the formula $\varphi$ in Example~\ref{ex:translation} 
can be written as:

{\small
\begin{verbatim}
(define-fun-rec s ((x Int)) Int (ite (<= x 0) 0 (s (- x 1))))
(assert (> (s x) 100))
(check-sat)
\end{verbatim}
}

When reading this input, 
\cvc adds the (annotated) quantified formula $\forallf{s} x.\;\, s( x ) \teq \lite( x \leq 0, 0, s( x-1 ) )$ to its list of assertions,
which after rewriting becomes $\forallf{s} x.\;\, \lite( x \leq 0, s( x ) \teq 0, s( x ) \teq s( x-1 ) )$.
If \cvc's finite model finding mode for recursive functions is enabled (using the command line parameter ``--fmf-fun"),
then it will replace its list of known assertions based on the conversion $\conv$ before checking for satisfiability.
%If the functions provided in the input are admissible, 
%then a ``satisfiable" response from the solver implies that a model exists for the original formula.
Accordingly, the solver will output an approximation of the interpretation for recursive function definitions,
that is, it may output the interpretation~(\ref{eq:approx-interp}) from Section~\ref{sec:encoding} for $s$.

Notice the option ``--fmf-fun" effectively instructs the SMT solver to assume that the functions provided by the user are admissible.
If this option is used for a set of functions definitions that is \emph{not} admissible, then the solver is not model-sound, that is, 
it may return ``satisfiable" for unsatisfiable inputs.
As such,
this option targets users who wish to determine the satisfiability of inputs that involving function definitions that are known to be consistent,
whereas it does not target users who wish to check the consistency of a set of function definitions.


\section{Related Work}

first-order model finders: SEM, MACE/Mace, Paradox, FMDarwin, iProver
  either direct search or reduction to SAT

Alloy (and Alloy idiom by Kuncak \& Jackson), Kodkod

The model finder for KIV based on Alloy

Refute
  * unsound handling of datatypes

Nitpick
  * inductive / coinductive predicates
  * quotient types

Leon

in smt : z3, cvc4 (briefly, text in main section gives detail)


\section{Conclusion}
\label{sec:conclusion}

  * future work
    * (co)inductive predicates?
    * quotient types??

{%\footnotesize
\def\ackname{Acknowledgment}
\paragraph{%\footnotesize
\ackname.}
We would like to thank Damien Busato-Gaston and Emmanouil Koukoutos for
providing the initial set of benchmarks used in the evaluation.
}



{
\bibliographystyle{abbrv}
\bibliography{bib}
}

\end{document}
